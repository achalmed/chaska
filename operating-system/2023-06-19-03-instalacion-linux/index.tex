\documentclass[
  jou,
  floatsintext,
  longtable,
  a4paper,
  nolmodern,
  notxfonts,
  notimes,
  colorlinks=true,linkcolor=blue,citecolor=blue,urlcolor=blue]{apa7}

\usepackage{amsmath}
\usepackage{amssymb}



\usepackage[bidi=default]{babel}
\babelprovide[main,import]{spanish}


\babelfont{rm}[,RawFeature={fallback=mainfontfallback}]{Latin Modern
Roman}
% get rid of language-specific shorthands (see #6817):
\let\LanguageShortHands\languageshorthands
\def\languageshorthands#1{}

\RequirePackage{longtable}
\RequirePackage{threeparttablex}

\makeatletter
\renewcommand{\paragraph}{\@startsection{paragraph}{4}{\parindent}%
	{0\baselineskip \@plus 0.2ex \@minus 0.2ex}%
	{-.5em}%
	{\normalfont\normalsize\bfseries\typesectitle}}

\renewcommand{\subparagraph}[1]{\@startsection{subparagraph}{5}{0.5em}%
	{0\baselineskip \@plus 0.2ex \@minus 0.2ex}%
	{-\z@\relax}%
	{\normalfont\normalsize\bfseries\itshape\hspace{\parindent}{#1}\textit{\addperi}}{\relax}}
\makeatother




\usepackage{longtable, booktabs, multirow, multicol, colortbl, hhline, caption, array, float, xpatch}
\usepackage{subcaption}


\renewcommand\thesubfigure{\Alph{subfigure}}
\setcounter{topnumber}{2}
\setcounter{bottomnumber}{2}
\setcounter{totalnumber}{4}
\renewcommand{\topfraction}{0.85}
\renewcommand{\bottomfraction}{0.85}
\renewcommand{\textfraction}{0.15}
\renewcommand{\floatpagefraction}{0.7}

\usepackage{tcolorbox}
\tcbuselibrary{listings,theorems, breakable, skins}
\usepackage{fontawesome5}

\definecolor{quarto-callout-color}{HTML}{909090}
\definecolor{quarto-callout-note-color}{HTML}{0758E5}
\definecolor{quarto-callout-important-color}{HTML}{CC1914}
\definecolor{quarto-callout-warning-color}{HTML}{EB9113}
\definecolor{quarto-callout-tip-color}{HTML}{00A047}
\definecolor{quarto-callout-caution-color}{HTML}{FC5300}
\definecolor{quarto-callout-color-frame}{HTML}{ACACAC}
\definecolor{quarto-callout-note-color-frame}{HTML}{4582EC}
\definecolor{quarto-callout-important-color-frame}{HTML}{D9534F}
\definecolor{quarto-callout-warning-color-frame}{HTML}{F0AD4E}
\definecolor{quarto-callout-tip-color-frame}{HTML}{02B875}
\definecolor{quarto-callout-caution-color-frame}{HTML}{FD7E14}

%\newlength\Oldarrayrulewidth
%\newlength\Oldtabcolsep


\usepackage{hyperref}




\providecommand{\tightlist}{%
  \setlength{\itemsep}{0pt}\setlength{\parskip}{0pt}}
\usepackage{longtable,booktabs,array}
\usepackage{calc} % for calculating minipage widths
% Correct order of tables after \paragraph or \subparagraph
\usepackage{etoolbox}
\makeatletter
\patchcmd\longtable{\par}{\if@noskipsec\mbox{}\fi\par}{}{}
\makeatother
% Allow footnotes in longtable head/foot
\IfFileExists{footnotehyper.sty}{\usepackage{footnotehyper}}{\usepackage{footnote}}
\makesavenoteenv{longtable}

\usepackage{graphicx}
\makeatletter
\newsavebox\pandoc@box
\newcommand*\pandocbounded[1]{% scales image to fit in text height/width
  \sbox\pandoc@box{#1}%
  \Gscale@div\@tempa{\textheight}{\dimexpr\ht\pandoc@box+\dp\pandoc@box\relax}%
  \Gscale@div\@tempb{\linewidth}{\wd\pandoc@box}%
  \ifdim\@tempb\p@<\@tempa\p@\let\@tempa\@tempb\fi% select the smaller of both
  \ifdim\@tempa\p@<\p@\scalebox{\@tempa}{\usebox\pandoc@box}%
  \else\usebox{\pandoc@box}%
  \fi%
}
% Set default figure placement to htbp
\def\fps@figure{htbp}
\makeatother







\usepackage{fontspec} 

\defaultfontfeatures{Scale=MatchLowercase}
\defaultfontfeatures[\rmfamily]{Ligatures=TeX,Scale=1}

  \setmainfont[,RawFeature={fallback=mainfontfallback}]{Latin Modern
Roman}




\title{Descarga, Instalación y Más de GNU/Linux: Descubre cómo
seleccionar hardware, descargar la imagen ISO y preparar los medios de
instalación. Exploraremos opciones para probar o instalar Linux en tu
equipo.}


\shorttitle{Editar}


\usepackage{etoolbox}



\ccoppy{\textcopyright~2023}



\author{Edison Achalma}



\affiliation{
{Escuela Profesional de Economía, Universidad Nacional de San Cristóbal
de Huamanga}}




\leftheader{Achalma}

\date{2023-06-19}


\abstract{Primer parrafo de abstrac }

\keywords{keyword1, keyword2}

\authornote{\par{\addORCIDlink{Edison Achalma}{0000-0001-6996-3364}} 
\par{ }
\par{   El autor no tiene conflictos de interés que revelar.    Los
roles de autor se clasificaron utilizando la taxonomía de roles de
colaborador (CRediT; https://credit.niso.org/) de la siguiente
manera:  Edison Achalma:   conceptualización, redacción}
\par{La correspondencia relativa a este artículo debe dirigirse a Edison
Achalma, Email: \href{mailto:elmer.achalma.09@unsch.edu.pe}{elmer.achalma.09@unsch.edu.pe}}
}

\usepackage{pbalance}
% \usepackage{float}
\makeatletter
\let\oldtpt\ThreePartTable
\let\endoldtpt\endThreePartTable
\def\ThreePartTable{\@ifnextchar[\ThreePartTable@i \ThreePartTable@ii}
\def\ThreePartTable@i[#1]{\begin{figure}[!htbp]
\onecolumn
\begin{minipage}{0.485\textwidth}
\oldtpt[#1]
}
\def\ThreePartTable@ii{\begin{figure}[!htbp]
\onecolumn
\begin{minipage}{0.48\textwidth}
\oldtpt
}
\def\endThreePartTable{
\endoldtpt
\end{minipage}
\twocolumn
\end{figure}}
\makeatother


\makeatletter
\let\endoldlt\endlongtable		
\def\endlongtable{
\hline
\endoldlt}
\makeatother

\newenvironment{twocolumntable}% environment name
{% begin code
\begin{table*}[!htbp]%
\onecolumn%
}%
{%
\twocolumn%
\end{table*}%
}% end code

\urlstyle{same}



\makeatletter
\@ifpackageloaded{caption}{}{\usepackage{caption}}
\AtBeginDocument{%
\ifdefined\contentsname
  \renewcommand*\contentsname{Tabla de contenidos}
\else
  \newcommand\contentsname{Tabla de contenidos}
\fi
\ifdefined\listfigurename
  \renewcommand*\listfigurename{List of Figures}
\else
  \newcommand\listfigurename{List of Figures}
\fi
\ifdefined\listtablename
  \renewcommand*\listtablename{List of Tables}
\else
  \newcommand\listtablename{List of Tables}
\fi
\ifdefined\figurename
  \renewcommand*\figurename{Figura}
\else
  \newcommand\figurename{Figura}
\fi
\ifdefined\tablename
  \renewcommand*\tablename{Tabla}
\else
  \newcommand\tablename{Tabla}
\fi
}
\@ifpackageloaded{float}{}{\usepackage{float}}
\floatstyle{ruled}
\@ifundefined{c@chapter}{\newfloat{codelisting}{h}{lop}}{\newfloat{codelisting}{h}{lop}[chapter]}
\floatname{codelisting}{Listing}
\newcommand*\listoflistings{\listof{codelisting}{List of Listings}}
\makeatother
\makeatletter
\makeatother
\makeatletter
\@ifpackageloaded{caption}{}{\usepackage{caption}}
\@ifpackageloaded{subcaption}{}{\usepackage{subcaption}}
\makeatother
\makeatletter
\@ifpackageloaded{fontawesome5}{}{\usepackage{fontawesome5}}
\makeatother

% From https://tex.stackexchange.com/a/645996/211326
%%% apa7 doesn't want to add appendix section titles in the toc
%%% let's make it do it
\makeatletter
\xpatchcmd{\appendix}
  {\par}
  {\addcontentsline{toc}{section}{\@currentlabelname}\par}
  {}{}
\makeatother

%% Disable longtable counter
%% https://tex.stackexchange.com/a/248395/211326

\usepackage{etoolbox}

\makeatletter
\patchcmd{\LT@caption}
  {\bgroup}
  {\bgroup\global\LTpatch@captiontrue}
  {}{}
\patchcmd{\longtable}
  {\par}
  {\par\global\LTpatch@captionfalse}
  {}{}
\apptocmd{\endlongtable}
  {\ifLTpatch@caption\else\addtocounter{table}{-1}\fi}
  {}{}
\newif\ifLTpatch@caption
\makeatother

\begin{document}

\maketitle

\hypertarget{toc}{}
\tableofcontents
\newpage
\section[Introduction]{Descarga, Instalación y Más de GNU/Linux}

\setcounter{secnumdepth}{5}

\setlength\LTleft{0pt}


¡Hola, lector!

Bienvenido a esta emocionante serie de introducción a Linux. Si estás
dando tus primeros pasos en este fascinante mundo, estás en el lugar
adecuado. Aquí, desgranaremos algunos aspectos fundamentales que debes
conocer, desde las distribuciones y entornos gráficos hasta las
aplicaciones y la administración del sistema.

En esta parte práctica, nos enfocaremos en los pasos necesarios para
descargar una distribución GNU/Linux y comenzar a probarla en tus
propios equipos. Te guiaré a través de diferentes etapas, desde la
selección del hardware hasta la descarga de la imagen ISO y la
preparación de los medios de instalación. Además, exploraremos diversas
opciones para probar o instalar Linux.

Si estás listo para embarcarte en esta emocionante aventura, acompáñame
en este recorrido. ¡Juntos descubriremos el poder y la versatilidad de
GNU/Linux!

\section{Seleccionar el Hardware}\label{seleccionar-el-hardware}

Antes de sumergirnos en la descarga e instalación de esta emocionante
plataforma, es importante seleccionar el hardware adecuado.

El hardware es el conjunto de componentes físicos de tu equipo, como el
procesador, la memoria RAM, el disco duro y otros dispositivos. Aunque
Linux es conocido por su capacidad para adaptarse a diferentes tipos de
hardware, es recomendable tener en cuenta algunas consideraciones.

En primer lugar, debes verificar los requisitos mínimos del sistema de
la distribución Linux que deseas instalar. Estos requisitos suelen estar
disponibles en el sitio web oficial de la distribución. Asegúrate de que
tu hardware cumpla con esos requisitos para disfrutar de un rendimiento
óptimo.

Además, es importante tener en cuenta el propósito de tu instalación de
Linux. ¿Es para un equipo de escritorio, un servidor o una máquina
virtual? Dependiendo de tus necesidades, es posible que debas considerar
diferentes características de hardware, como la capacidad de
almacenamiento, la potencia de procesamiento y la compatibilidad de los
controladores.

Recuerda que Linux es conocido por su compatibilidad con una amplia gama
de hardware, incluidos equipos más antiguos. Así que, incluso si no
tienes lo último en tecnología, ¡no te preocupes! Es probable que Linux
funcione sin problemas en tu equipo.

Antes de pasar al siguiente paso, te recomendaría hacer una lista de las
especificaciones de tu hardware actual para compararlas con los
requisitos mínimos del sistema de la distribución Linux que elijas. Esto
te ayudará a tomar una decisión informada y asegurarte de que todo
funcione sin problemas.

\section{Descargar Imagen ISO de tu Distribución
Preferida}\label{descargar-imagen-iso-de-tu-distribuciuxf3n-preferida}

Ahora que hemos seleccionado el hardware adecuado, es hora de descargar
la imagen ISO de la distribución Linux que más te llame la atención. La
imagen ISO es un archivo que contiene todos los datos necesarios para
instalar el sistema operativo en tu equipo.

El primer paso es visitar el sitio web oficial de la distribución Linux
que deseas probar. Allí encontrarás una variedad de opciones para
descargar la imagen ISO. ¡Pero cuidado! Asegúrate de elegir la versión
correcta de acuerdo con tu arquitectura de hardware (generalmente 32
bits o 64 bits) y tus preferencias (como el entorno gráfico).

A continuación de dejo los enlaces a las paginas de descarga de las
ultimas versiones de algunas de las distribuciones más populares:

\begin{itemize}
\item
  \href{https://ubuntu.com/download/desktop}{Descarga Ubuntu 22.04 LTS}
\item
  \href{https://www.linuxmint.com/download.php}{Descarga Linux Mint 21}
\item
  \href{https://www.debian.org/download}{Descarga Debian 11 Bullseye}
\item
  \href{https://get.opensuse.org/leap/15.4/}{Descarga openSUSE Leap
  15.4}
\item
  \href{https://getfedora.org/es/workstation/download/}{Descargar Fedora
  36 Workstation}
\end{itemize}

Una vez que hayas seleccionado la versión adecuada, simplemente haz clic
en el enlace de descarga correspondiente. Dependiendo del tamaño del
archivo y la velocidad de tu conexión a Internet, la descarga puede
llevar algunos minutos. Aprovecha este tiempo para preparar una buena
taza de café o té, y relájate mientras se completa la descarga.

Una vez que la descarga haya finalizado, ¡felicitaciones! Ahora tienes
la imagen ISO de tu distribución Linux preferida lista para usar. Pero
antes de seguir adelante, te recomendaría verificar la integridad del
archivo. Algunos sitios web proporcionan sumas de verificación o firmas
digitales que te permiten asegurarte de que el archivo se descargó
correctamente y no se haya corrompido en el proceso.

Si todo está en orden, estás listo para el siguiente paso emocionante:
preparar los medios de instalación. En nuestro próximo fragmento,
exploraremos cómo convertir la imagen ISO en un medio de instalación, ya
sea una unidad USB o un DVD.

\section{Preparar los Medios de
Instalación}\label{preparar-los-medios-de-instalaciuxf3n}

¡Excelente, explorador de Linux! Ahora que tienes la imagen ISO de tu
distribución preferida, es hora de convertirla en un medio de
instalación. Esto te permitirá instalar Linux en tu equipo y comenzar a
explorar sus maravillas.

La forma más común de preparar los medios de instalación es crear un
dispositivo USB de arranque. Para hacer esto, necesitarás una unidad USB
vacía con capacidad suficiente para albergar la imagen ISO. Asegúrate de
hacer una copia de seguridad de cualquier dato importante que haya en la
unidad USB, ya que se eliminará durante el proceso.

Para crear el dispositivo USB de arranque, existen varias herramientas
disponibles. Una de las más populares y fáciles de usar es ``Etcher''.
Puedes descargarlo desde su sitio web oficial e instalarlo en tu sistema
operativo actual.

Una vez que hayas instalado Etcher, ábrelo y verás una interfaz sencilla
y amigable. En primer lugar, selecciona la imagen ISO que descargaste
haciendo clic en el botón ``Seleccionar imagen''. Luego, elige la unidad
USB en la que deseas crear el dispositivo de arranque.

Una vez que hayas hecho todas las selecciones, haz clic en el botón
``Flash!'' y deja que Etcher haga su magia. Ten en cuenta que este
proceso puede tardar unos minutos, así que ten paciencia.

Una vez completado el proceso, tendrás tu dispositivo USB de arranque
listo para usar. Ahora estás un paso más cerca de sumergirte en el mundo
de Linux.

\subsection{Opciones para Probar o Instalar
Linux}\label{opciones-para-probar-o-instalar-linux}

¡Es hora de decidir cómo quieres probar o instalar Linux en tu equipo,
intrépido lector! Afortunadamente, tienes varias opciones disponibles
que se adaptan a tus necesidades y preferencias.

La primera opción es probar Linux sin realizar una instalación completa
en tu equipo. Esto se conoce como \textbf{``modo Live''.} Es ideal si
quieres explorar Linux sin hacer cambios permanentes en tu sistema
operativo actual. Simplemente inserta el dispositivo USB de arranque que
creaste en el paso anterior, reinicia tu equipo y selecciona la opción
``Probar Linux'' en el menú de arranque. Esto cargará Linux desde el
dispositivo USB y podrás experimentar su funcionalidad y
características. ¡Es como hacer una prueba de manejo antes de decidirte!

Si te sientes cómodo y deseas instalar Linux de forma permanente en tu
equipo, también tienes esa opción. Reinicia tu equipo con el dispositivo
USB de arranque insertado y elige la opción \textbf{``Instalar Linux''}
en el menú de arranque. Esto iniciará el asistente de instalación, donde
podrás seguir los pasos sencillos para instalar Linux en tu disco duro.
Recuerda que durante este proceso, se te pedirá que tomes decisiones,
como la partición del disco y la configuración de usuario. ¡Pero no te
preocupes, estaré aquí para guiarte!

Una tercera opción, si no deseas hacer cambios en tu disco duro
principal, es utilizar una \textbf{máquina virtual.} Una máquina virtual
es un entorno virtualizado en tu sistema operativo actual donde puedes
ejecutar Linux como si fuera un programa. Existen varias aplicaciones de
máquina virtual disponibles, como VirtualBox o VMware Player.
Simplemente descarga e instala una de estas aplicaciones, crea una nueva
máquina virtual, selecciona la imagen ISO de Linux y ¡voilà!, tendrás
Linux funcionando dentro de tu sistema operativo actual.

¡Ya casi estás allí, valiente explorador! Estas son las opciones
principales para probar o instalar Linux en tu equipo. Elige la que
mejor se adapte a tus necesidades y nivel de comodidad. ¡No te
preocupes, cualquier opción que elijas, te espera un emocionante viaje
hacia el mundo de Linux!

\section{Publicaciones Similares}\label{publicaciones-similares}

Si te interesó este artículo, te recomendamos que explores otros blogs y
recursos relacionados que pueden ampliar tus conocimientos. Aquí te dejo
algunas sugerencias:

\begin{enumerate}
\def\labelenumi{\arabic{enumi}.}
\tightlist
\item
  \href{https://achalmaedison.netlify.app/tecnologia-seguridad/operating-system/2017-05-21-comandos-de-informacion-windows/index.pdf}{\faIcon{file-pdf}}
  \href{https://achalmaedison.netlify.app/tecnologia-seguridad/operating-system/2017-05-21-comandos-de-informacion-windows}{Comandos
  De Informacion Windows}
\item
  \href{https://achalmaedison.netlify.app/tecnologia-seguridad/operating-system/2019-06-19-adb/index.pdf}{\faIcon{file-pdf}}
  \href{https://achalmaedison.netlify.app/tecnologia-seguridad/operating-system/2019-06-19-adb}{Adb}
\item
  \href{https://achalmaedison.netlify.app/tecnologia-seguridad/operating-system/2021-08-17-limpieza-y-optimizacion-de-pc/index.pdf}{\faIcon{file-pdf}}
  \href{https://achalmaedison.netlify.app/tecnologia-seguridad/operating-system/2021-08-17-limpieza-y-optimizacion-de-pc}{Limpieza
  Y Optimizacion De Pc}
\item
  \href{https://achalmaedison.netlify.app/tecnologia-seguridad/operating-system/2021-10-21-usando-apk-en-windown-11/index.pdf}{\faIcon{file-pdf}}
  \href{https://achalmaedison.netlify.app/tecnologia-seguridad/operating-system/2021-10-21-usando-apk-en-windown-11}{Usando
  Apk En Windown 11}
\item
  \href{https://achalmaedison.netlify.app/tecnologia-seguridad/operating-system/2022-05-12-gestionar-versiones-de-jdk-en-kubuntu/index.pdf}{\faIcon{file-pdf}}
  \href{https://achalmaedison.netlify.app/tecnologia-seguridad/operating-system/2022-05-12-gestionar-versiones-de-jdk-en-kubuntu}{Gestionar
  Versiones De Jdk En Kubuntu}
\item
  \href{https://achalmaedison.netlify.app/tecnologia-seguridad/operating-system/2022-07-21-instalar-tor-browser/index.pdf}{\faIcon{file-pdf}}
  \href{https://achalmaedison.netlify.app/tecnologia-seguridad/operating-system/2022-07-21-instalar-tor-browser}{Instalar
  Tor Browser}
\item
  \href{https://achalmaedison.netlify.app/tecnologia-seguridad/operating-system/2022-08-14-crear-enlaces-duros-o-hard-link-en-linux/index.pdf}{\faIcon{file-pdf}}
  \href{https://achalmaedison.netlify.app/tecnologia-seguridad/operating-system/2022-08-14-crear-enlaces-duros-o-hard-link-en-linux}{Crear
  Enlaces Duros O Hard Link En Linux}
\item
  \href{https://achalmaedison.netlify.app/tecnologia-seguridad/operating-system/2022-09-27-comandos-vim/index.pdf}{\faIcon{file-pdf}}
  \href{https://achalmaedison.netlify.app/tecnologia-seguridad/operating-system/2022-09-27-comandos-vim}{Comandos
  Vim}
\item
  \href{https://achalmaedison.netlify.app/tecnologia-seguridad/operating-system/2023-02-16-guia-de-git-y-github/index.pdf}{\faIcon{file-pdf}}
  \href{https://achalmaedison.netlify.app/tecnologia-seguridad/operating-system/2023-02-16-guia-de-git-y-github}{Guia
  De Git Y Github}
\item
  \href{https://achalmaedison.netlify.app/tecnologia-seguridad/operating-system/2023-05-02-00-primeros-pasos-en-linux/index.pdf}{\faIcon{file-pdf}}
  \href{https://achalmaedison.netlify.app/tecnologia-seguridad/operating-system/2023-05-02-00-primeros-pasos-en-linux}{00
  Primeros Pasos En Linux}
\item
  \href{https://achalmaedison.netlify.app/tecnologia-seguridad/operating-system/2023-06-17-01-introduccion-linux/index.pdf}{\faIcon{file-pdf}}
  \href{https://achalmaedison.netlify.app/tecnologia-seguridad/operating-system/2023-06-17-01-introduccion-linux}{01
  Introduccion Linux}
\item
  \href{https://achalmaedison.netlify.app/tecnologia-seguridad/operating-system/2023-06-18-02-distribuciones-linux/index.pdf}{\faIcon{file-pdf}}
  \href{https://achalmaedison.netlify.app/tecnologia-seguridad/operating-system/2023-06-18-02-distribuciones-linux}{02
  Distribuciones Linux}
\item
  \href{https://achalmaedison.netlify.app/tecnologia-seguridad/operating-system/2023-06-19-03-instalacion-linux/index.pdf}{\faIcon{file-pdf}}
  \href{https://achalmaedison.netlify.app/tecnologia-seguridad/operating-system/2023-06-19-03-instalacion-linux}{03
  Instalacion Linux}
\item
  \href{https://achalmaedison.netlify.app/tecnologia-seguridad/operating-system/2023-06-20-04-administracion-particiones-volumenes/index.pdf}{\faIcon{file-pdf}}
  \href{https://achalmaedison.netlify.app/tecnologia-seguridad/operating-system/2023-06-20-04-administracion-particiones-volumenes}{04
  Administracion Particiones Volumenes}
\item
  \href{https://achalmaedison.netlify.app/tecnologia-seguridad/operating-system/2023-07-01-atajos-de-teclado-y-comandos-para-usar-vim/index.pdf}{\faIcon{file-pdf}}
  \href{https://achalmaedison.netlify.app/tecnologia-seguridad/operating-system/2023-07-01-atajos-de-teclado-y-comandos-para-usar-vim}{Atajos
  De Teclado Y Comandos Para Usar Vim}
\item
  \href{https://achalmaedison.netlify.app/tecnologia-seguridad/operating-system/2024-07-15-instalando-specitify/index.pdf}{\faIcon{file-pdf}}
  \href{https://achalmaedison.netlify.app/tecnologia-seguridad/operating-system/2024-07-15-instalando-specitify}{Instalando
  Specitify}
\end{enumerate}

Esperamos que encuentres estas publicaciones igualmente interesantes y
útiles. ¡Disfruta de la lectura!






\end{document}
