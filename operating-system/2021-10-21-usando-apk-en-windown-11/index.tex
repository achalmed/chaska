\documentclass[
  doc,
  floatsintext,
  longtable,
  a4paper,
  nolmodern,
  notxfonts,
  notimes,
  colorlinks=true,linkcolor=blue,citecolor=blue,urlcolor=blue]{apa7}

\usepackage{amsmath}
\usepackage{amssymb}

\geometry{inner=1in, outer=1in}
\fancyhfoffset[LE,RO]{0cm}


\usepackage[bidi=default]{babel}
\babelprovide[main,import]{spanish}


\babelfont{rm}[,RawFeature={fallback=mainfontfallback}]{Latin Modern
Roman}
% get rid of language-specific shorthands (see #6817):
\let\LanguageShortHands\languageshorthands
\def\languageshorthands#1{}

\RequirePackage{longtable}
\RequirePackage{threeparttablex}

\makeatletter
\renewcommand{\paragraph}{\@startsection{paragraph}{4}{\parindent}%
	{0\baselineskip \@plus 0.2ex \@minus 0.2ex}%
	{-.5em}%
	{\normalfont\normalsize\bfseries\typesectitle}}

\renewcommand{\subparagraph}[1]{\@startsection{subparagraph}{5}{0.5em}%
	{0\baselineskip \@plus 0.2ex \@minus 0.2ex}%
	{-\z@\relax}%
	{\normalfont\normalsize\bfseries\itshape\hspace{\parindent}{#1}\textit{\addperi}}{\relax}}
\makeatother




\usepackage{longtable, booktabs, multirow, multicol, colortbl, hhline, caption, array, float, xpatch}
\usepackage{subcaption}


\renewcommand\thesubfigure{\Alph{subfigure}}
\setcounter{topnumber}{2}
\setcounter{bottomnumber}{2}
\setcounter{totalnumber}{4}
\renewcommand{\topfraction}{0.85}
\renewcommand{\bottomfraction}{0.85}
\renewcommand{\textfraction}{0.15}
\renewcommand{\floatpagefraction}{0.7}

\usepackage{tcolorbox}
\tcbuselibrary{listings,theorems, breakable, skins}
\usepackage{fontawesome5}

\definecolor{quarto-callout-color}{HTML}{909090}
\definecolor{quarto-callout-note-color}{HTML}{0758E5}
\definecolor{quarto-callout-important-color}{HTML}{CC1914}
\definecolor{quarto-callout-warning-color}{HTML}{EB9113}
\definecolor{quarto-callout-tip-color}{HTML}{00A047}
\definecolor{quarto-callout-caution-color}{HTML}{FC5300}
\definecolor{quarto-callout-color-frame}{HTML}{ACACAC}
\definecolor{quarto-callout-note-color-frame}{HTML}{4582EC}
\definecolor{quarto-callout-important-color-frame}{HTML}{D9534F}
\definecolor{quarto-callout-warning-color-frame}{HTML}{F0AD4E}
\definecolor{quarto-callout-tip-color-frame}{HTML}{02B875}
\definecolor{quarto-callout-caution-color-frame}{HTML}{FD7E14}

%\newlength\Oldarrayrulewidth
%\newlength\Oldtabcolsep


\usepackage{hyperref}



\usepackage{color}
\usepackage{fancyvrb}
\newcommand{\VerbBar}{|}
\newcommand{\VERB}{\Verb[commandchars=\\\{\}]}
\DefineVerbatimEnvironment{Highlighting}{Verbatim}{commandchars=\\\{\}}
% Add ',fontsize=\small' for more characters per line
\usepackage{framed}
\definecolor{shadecolor}{RGB}{241,243,245}
\newenvironment{Shaded}{\begin{snugshade}}{\end{snugshade}}
\newcommand{\AlertTok}[1]{\textcolor[rgb]{0.68,0.00,0.00}{#1}}
\newcommand{\AnnotationTok}[1]{\textcolor[rgb]{0.37,0.37,0.37}{#1}}
\newcommand{\AttributeTok}[1]{\textcolor[rgb]{0.40,0.45,0.13}{#1}}
\newcommand{\BaseNTok}[1]{\textcolor[rgb]{0.68,0.00,0.00}{#1}}
\newcommand{\BuiltInTok}[1]{\textcolor[rgb]{0.00,0.23,0.31}{#1}}
\newcommand{\CharTok}[1]{\textcolor[rgb]{0.13,0.47,0.30}{#1}}
\newcommand{\CommentTok}[1]{\textcolor[rgb]{0.37,0.37,0.37}{#1}}
\newcommand{\CommentVarTok}[1]{\textcolor[rgb]{0.37,0.37,0.37}{\textit{#1}}}
\newcommand{\ConstantTok}[1]{\textcolor[rgb]{0.56,0.35,0.01}{#1}}
\newcommand{\ControlFlowTok}[1]{\textcolor[rgb]{0.00,0.23,0.31}{\textbf{#1}}}
\newcommand{\DataTypeTok}[1]{\textcolor[rgb]{0.68,0.00,0.00}{#1}}
\newcommand{\DecValTok}[1]{\textcolor[rgb]{0.68,0.00,0.00}{#1}}
\newcommand{\DocumentationTok}[1]{\textcolor[rgb]{0.37,0.37,0.37}{\textit{#1}}}
\newcommand{\ErrorTok}[1]{\textcolor[rgb]{0.68,0.00,0.00}{#1}}
\newcommand{\ExtensionTok}[1]{\textcolor[rgb]{0.00,0.23,0.31}{#1}}
\newcommand{\FloatTok}[1]{\textcolor[rgb]{0.68,0.00,0.00}{#1}}
\newcommand{\FunctionTok}[1]{\textcolor[rgb]{0.28,0.35,0.67}{#1}}
\newcommand{\ImportTok}[1]{\textcolor[rgb]{0.00,0.46,0.62}{#1}}
\newcommand{\InformationTok}[1]{\textcolor[rgb]{0.37,0.37,0.37}{#1}}
\newcommand{\KeywordTok}[1]{\textcolor[rgb]{0.00,0.23,0.31}{\textbf{#1}}}
\newcommand{\NormalTok}[1]{\textcolor[rgb]{0.00,0.23,0.31}{#1}}
\newcommand{\OperatorTok}[1]{\textcolor[rgb]{0.37,0.37,0.37}{#1}}
\newcommand{\OtherTok}[1]{\textcolor[rgb]{0.00,0.23,0.31}{#1}}
\newcommand{\PreprocessorTok}[1]{\textcolor[rgb]{0.68,0.00,0.00}{#1}}
\newcommand{\RegionMarkerTok}[1]{\textcolor[rgb]{0.00,0.23,0.31}{#1}}
\newcommand{\SpecialCharTok}[1]{\textcolor[rgb]{0.37,0.37,0.37}{#1}}
\newcommand{\SpecialStringTok}[1]{\textcolor[rgb]{0.13,0.47,0.30}{#1}}
\newcommand{\StringTok}[1]{\textcolor[rgb]{0.13,0.47,0.30}{#1}}
\newcommand{\VariableTok}[1]{\textcolor[rgb]{0.07,0.07,0.07}{#1}}
\newcommand{\VerbatimStringTok}[1]{\textcolor[rgb]{0.13,0.47,0.30}{#1}}
\newcommand{\WarningTok}[1]{\textcolor[rgb]{0.37,0.37,0.37}{\textit{#1}}}

\providecommand{\tightlist}{%
  \setlength{\itemsep}{0pt}\setlength{\parskip}{0pt}}
\usepackage{longtable,booktabs,array}
\usepackage{calc} % for calculating minipage widths
% Correct order of tables after \paragraph or \subparagraph
\usepackage{etoolbox}
\makeatletter
\patchcmd\longtable{\par}{\if@noskipsec\mbox{}\fi\par}{}{}
\makeatother
% Allow footnotes in longtable head/foot
\IfFileExists{footnotehyper.sty}{\usepackage{footnotehyper}}{\usepackage{footnote}}
\makesavenoteenv{longtable}

\usepackage{graphicx}
\makeatletter
\newsavebox\pandoc@box
\newcommand*\pandocbounded[1]{% scales image to fit in text height/width
  \sbox\pandoc@box{#1}%
  \Gscale@div\@tempa{\textheight}{\dimexpr\ht\pandoc@box+\dp\pandoc@box\relax}%
  \Gscale@div\@tempb{\linewidth}{\wd\pandoc@box}%
  \ifdim\@tempb\p@<\@tempa\p@\let\@tempa\@tempb\fi% select the smaller of both
  \ifdim\@tempa\p@<\p@\scalebox{\@tempa}{\usebox\pandoc@box}%
  \else\usebox{\pandoc@box}%
  \fi%
}
% Set default figure placement to htbp
\def\fps@figure{htbp}
\makeatother







\usepackage{fontspec} 

\defaultfontfeatures{Scale=MatchLowercase}
\defaultfontfeatures[\rmfamily]{Ligatures=TeX,Scale=1}

  \setmainfont[,RawFeature={fallback=mainfontfallback}]{Latin Modern
Roman}




\title{Cómo usar APK en Windows 11 una guía paso a paso: Aprende a
instalar y ejecutar aplicaciones de Android en tu PC con Windows 11}


\shorttitle{Editar}


\usepackage{etoolbox}



\ccoppy{\textcopyright~2021}



\author{Edison Achalma}



\affiliation{
{Escuela Profesional de Economía, Universidad Nacional de San Cristóbal
de Huamanga}}




\leftheader{Achalma}

\date{2021-10-21}


\abstract{Primer parrafo de abstrac }

\keywords{keyword1, keyword2}

\authornote{\par{\addORCIDlink{Edison Achalma}{0000-0001-6996-3364}} 
\par{ }
\par{   El autor no tiene conflictos de interés que revelar.    Los
roles de autor se clasificaron utilizando la taxonomía de roles de
colaborador (CRediT; https://credit.niso.org/) de la siguiente
manera:  Edison Achalma:   conceptualización, redacción}
\par{La correspondencia relativa a este artículo debe dirigirse a Edison
Achalma, Email: \href{mailto:elmer.achalma.09@unsch.edu.pe}{elmer.achalma.09@unsch.edu.pe}}
}

\makeatletter
\let\endoldlt\endlongtable
\def\endlongtable{
\hline
\endoldlt
}
\makeatother

\urlstyle{same}



\makeatletter
\@ifpackageloaded{caption}{}{\usepackage{caption}}
\AtBeginDocument{%
\ifdefined\contentsname
  \renewcommand*\contentsname{Tabla de contenidos}
\else
  \newcommand\contentsname{Tabla de contenidos}
\fi
\ifdefined\listfigurename
  \renewcommand*\listfigurename{List of Figures}
\else
  \newcommand\listfigurename{List of Figures}
\fi
\ifdefined\listtablename
  \renewcommand*\listtablename{List of Tables}
\else
  \newcommand\listtablename{List of Tables}
\fi
\ifdefined\figurename
  \renewcommand*\figurename{Figura}
\else
  \newcommand\figurename{Figura}
\fi
\ifdefined\tablename
  \renewcommand*\tablename{Tabla}
\else
  \newcommand\tablename{Tabla}
\fi
}
\@ifpackageloaded{float}{}{\usepackage{float}}
\floatstyle{ruled}
\@ifundefined{c@chapter}{\newfloat{codelisting}{h}{lop}}{\newfloat{codelisting}{h}{lop}[chapter]}
\floatname{codelisting}{Listing}
\newcommand*\listoflistings{\listof{codelisting}{List of Listings}}
\makeatother
\makeatletter
\makeatother
\makeatletter
\@ifpackageloaded{caption}{}{\usepackage{caption}}
\@ifpackageloaded{subcaption}{}{\usepackage{subcaption}}
\makeatother
\makeatletter
\@ifpackageloaded{fontawesome5}{}{\usepackage{fontawesome5}}
\makeatother

% From https://tex.stackexchange.com/a/645996/211326
%%% apa7 doesn't want to add appendix section titles in the toc
%%% let's make it do it
\makeatletter
\xpatchcmd{\appendix}
  {\par}
  {\addcontentsline{toc}{section}{\@currentlabelname}\par}
  {}{}
\makeatother

%% Disable longtable counter
%% https://tex.stackexchange.com/a/248395/211326

\usepackage{etoolbox}

\makeatletter
\patchcmd{\LT@caption}
  {\bgroup}
  {\bgroup\global\LTpatch@captiontrue}
  {}{}
\patchcmd{\longtable}
  {\par}
  {\par\global\LTpatch@captionfalse}
  {}{}
\apptocmd{\endlongtable}
  {\ifLTpatch@caption\else\addtocounter{table}{-1}\fi}
  {}{}
\newif\ifLTpatch@caption
\makeatother

\begin{document}

\maketitle

\hypertarget{toc}{}
\tableofcontents
\newpage
\section[Introduction]{Cómo usar APK en Windows 11 una guía paso a paso}

\setcounter{secnumdepth}{5}

\setlength\LTleft{0pt}


\section{Windows 11: Cómo descargar APK usando el Subsistema de Windows
para Android y
ADB}\label{windows-11-cuxf3mo-descargar-apk-usando-el-subsistema-de-windows-para-android-y-adb}

Aquí te explicamos cómo descargar un archivo APK para instalar una
aplicación de Android en tu PC con Windows 11 usando el Subsistema de
Windows para Android. Puedes instalar el Subsistema de Windows para
Android manualmente en tu PC con su archivo Msixbundle siguiendo nuestra
\href{https://nerdschalk.com/android-apps-on-windows-11-dev-channel-how-to-install-windows-subsystem-for-android-manually-with-msixbundle/}{guía
aquí}.

\subsection{Paso 1: Habilitar el modo de desarrollador en el subsistema
de
Windows}\label{paso-1-habilitar-el-modo-de-desarrollador-en-el-subsistema-de-windows}

\begin{enumerate}
\def\labelenumi{\arabic{enumi}.}
\tightlist
\item
  Instala el
  \href{https://nerdschalk.com/android-apps-on-windows-11-dev-channel-how-to-install-windows-subsystem-for-android-manually-with-msixbundle/}{Subsistema
  de Windows para Android}.
\item
  Abre la aplicación `Subsistema de Windows para Android' en tu PC. Para
  ello, presiona la tecla \textbf{Windows} y busca \textbf{Subsistema de
  Windows para Android}.
\item
  Haz clic en la aplicación para abrirla.
\item
  Dentro de la aplicación, activa el \textbf{Modo Desarrollador}.
\end{enumerate}

\subsection{Paso 2: Instalar las herramientas de la plataforma
SDK}\label{paso-2-instalar-las-herramientas-de-la-plataforma-sdk}

\begin{enumerate}
\def\labelenumi{\arabic{enumi}.}
\tightlist
\item
  Visita la página de herramientas de la plataforma SDK de Google
  \href{https://developer.android.com/studio/releases/platform-tools.html}{aquí}.
\item
  Descarga \textbf{SDK Platform-Tools para Windows}.
\item
  Acepta los términos y condiciones y haz clic en el botón de descarga.
\item
  Se descargará un archivo ZIP llamado
  \textbf{platform-tools\_rXX.X.X-windows.zip} (la versión puede
  variar).
\item
  Crea una carpeta separada en el Explorador de Windows, por ejemplo,
  \texttt{C:\textbackslash{}Plataforma-Tools}.
\item
  Mueve el archivo ZIP descargado a esta carpeta.
\item
  Haz clic derecho en el archivo y selecciona \textbf{Extraer todo},
  luego haz clic en \textbf{Extraer}.
\item
  Abre la carpeta \texttt{platform-tools}, donde encontrarás
  \texttt{adb.exe} y otros archivos.
\end{enumerate}

\subsection{Paso 3: Instalar la aplicación de
Android}\label{paso-3-instalar-la-aplicaciuxf3n-de-android}

\begin{enumerate}
\def\labelenumi{\arabic{enumi}.}
\item
  Abre la carpeta \textbf{platform-tools}.
\item
  Haz clic en la barra de direcciones, escribe \textbf{\texttt{cmd}} y
  presiona \textbf{Enter}.
\item
  Se abrirá una ventana de comandos en la ubicación de la carpeta
  \textbf{platform-tools}.
\item
  Descarga el archivo APK de la aplicación de Android que deseas
  instalar.

  \begin{itemize}
  \tightlist
  \item
    Por ejemplo, para instalar Snapchat, busca \textbf{Snapchat APK} en
    Google y descarga el archivo de una fuente confiable.
  \item
    Renombra el archivo a algo simple, como \texttt{snapchat.apk}, y
    muévelo a la carpeta \textbf{platform-tools}.
  \end{itemize}
\item
  Abre el \textbf{Subsistema de Windows para Android} y copia la
  dirección \textbf{IP} en la opción de \textbf{Modo Desarrollador}.
\item
  En la ventana de comandos, ejecuta el siguiente comando:

\begin{Shaded}
\begin{Highlighting}[]
\ExtensionTok{adb.exe}\NormalTok{ connect }\PreprocessorTok{[}\SpecialStringTok{DIRECCIÓN\_IP}\PreprocessorTok{]}
\end{Highlighting}
\end{Shaded}

  \textbf{Ejemplo:}

\begin{Shaded}
\begin{Highlighting}[]
\ExtensionTok{adb.exe}\NormalTok{ connect 127.0.0.1:12345}
\end{Highlighting}
\end{Shaded}
\item
  Luego, instala la aplicación ejecutando:

\begin{Shaded}
\begin{Highlighting}[]
\ExtensionTok{adb.exe}\NormalTok{ install }\PreprocessorTok{[}\SpecialStringTok{NOMBRE\_DEL\_APK}\PreprocessorTok{]}
\end{Highlighting}
\end{Shaded}

  \textbf{Ejemplo:}

\begin{Shaded}
\begin{Highlighting}[]
\ExtensionTok{adb.exe}\NormalTok{ install snapchat.apk}
\end{Highlighting}
\end{Shaded}
\item
  Cuando la instalación finalice, verás el mensaje \textbf{Success}.
\item
  Cierra la ventana de comandos.
\item
  Abre la aplicación en tu PC escribiendo su nombre en el menú Inicio
  (por ejemplo, \textbf{Snapchat}).
\end{enumerate}

\subsection{Cargar APK automáticamente con doble
clic}\label{cargar-apk-automuxe1ticamente-con-doble-clic}

Si prefieres instalar APKs con un doble clic en lugar de usar comandos
ADB, puedes configurarlo siguiendo nuestra guía \hyperref[]{aquí}.

\section{Publicaciones Similares}\label{publicaciones-similares}

Si te interesó este artículo, te recomendamos que explores otros blogs y
recursos relacionados que pueden ampliar tus conocimientos. Aquí te dejo
algunas sugerencias:

\begin{enumerate}
\def\labelenumi{\arabic{enumi}.}
\tightlist
\item
  \href{https://achalmaedison.netlify.app/tecnologia-seguridad/operating-system/2017-05-21-comandos-de-informacion-windows/index.pdf}{\faIcon{file-pdf}}
  \href{https://achalmaedison.netlify.app/tecnologia-seguridad/operating-system/2017-05-21-comandos-de-informacion-windows}{Comandos
  De Informacion Windows}
\item
  \href{https://achalmaedison.netlify.app/tecnologia-seguridad/operating-system/2019-06-19-adb/index.pdf}{\faIcon{file-pdf}}
  \href{https://achalmaedison.netlify.app/tecnologia-seguridad/operating-system/2019-06-19-adb}{Adb}
\item
  \href{https://achalmaedison.netlify.app/tecnologia-seguridad/operating-system/2021-08-17-limpieza-y-optimizacion-de-pc/index.pdf}{\faIcon{file-pdf}}
  \href{https://achalmaedison.netlify.app/tecnologia-seguridad/operating-system/2021-08-17-limpieza-y-optimizacion-de-pc}{Limpieza
  Y Optimizacion De Pc}
\item
  \href{https://achalmaedison.netlify.app/tecnologia-seguridad/operating-system/2021-10-21-usando-apk-en-windown-11/index.pdf}{\faIcon{file-pdf}}
  \href{https://achalmaedison.netlify.app/tecnologia-seguridad/operating-system/2021-10-21-usando-apk-en-windown-11}{Usando
  Apk En Windown 11}
\item
  \href{https://achalmaedison.netlify.app/tecnologia-seguridad/operating-system/2022-05-12-gestionar-versiones-de-jdk-en-kubuntu/index.pdf}{\faIcon{file-pdf}}
  \href{https://achalmaedison.netlify.app/tecnologia-seguridad/operating-system/2022-05-12-gestionar-versiones-de-jdk-en-kubuntu}{Gestionar
  Versiones De Jdk En Kubuntu}
\item
  \href{https://achalmaedison.netlify.app/tecnologia-seguridad/operating-system/2022-07-21-instalar-tor-browser/index.pdf}{\faIcon{file-pdf}}
  \href{https://achalmaedison.netlify.app/tecnologia-seguridad/operating-system/2022-07-21-instalar-tor-browser}{Instalar
  Tor Browser}
\item
  \href{https://achalmaedison.netlify.app/tecnologia-seguridad/operating-system/2022-08-14-crear-enlaces-duros-o-hard-link-en-linux/index.pdf}{\faIcon{file-pdf}}
  \href{https://achalmaedison.netlify.app/tecnologia-seguridad/operating-system/2022-08-14-crear-enlaces-duros-o-hard-link-en-linux}{Crear
  Enlaces Duros O Hard Link En Linux}
\item
  \href{https://achalmaedison.netlify.app/tecnologia-seguridad/operating-system/2022-09-27-comandos-vim/index.pdf}{\faIcon{file-pdf}}
  \href{https://achalmaedison.netlify.app/tecnologia-seguridad/operating-system/2022-09-27-comandos-vim}{Comandos
  Vim}
\item
  \href{https://achalmaedison.netlify.app/tecnologia-seguridad/operating-system/2023-02-16-guia-de-git-y-github/index.pdf}{\faIcon{file-pdf}}
  \href{https://achalmaedison.netlify.app/tecnologia-seguridad/operating-system/2023-02-16-guia-de-git-y-github}{Guia
  De Git Y Github}
\item
  \href{https://achalmaedison.netlify.app/tecnologia-seguridad/operating-system/2023-05-02-00-primeros-pasos-en-linux/index.pdf}{\faIcon{file-pdf}}
  \href{https://achalmaedison.netlify.app/tecnologia-seguridad/operating-system/2023-05-02-00-primeros-pasos-en-linux}{00
  Primeros Pasos En Linux}
\item
  \href{https://achalmaedison.netlify.app/tecnologia-seguridad/operating-system/2023-06-17-01-introduccion-linux/index.pdf}{\faIcon{file-pdf}}
  \href{https://achalmaedison.netlify.app/tecnologia-seguridad/operating-system/2023-06-17-01-introduccion-linux}{01
  Introduccion Linux}
\item
  \href{https://achalmaedison.netlify.app/tecnologia-seguridad/operating-system/2023-06-18-02-distribuciones-linux/index.pdf}{\faIcon{file-pdf}}
  \href{https://achalmaedison.netlify.app/tecnologia-seguridad/operating-system/2023-06-18-02-distribuciones-linux}{02
  Distribuciones Linux}
\item
  \href{https://achalmaedison.netlify.app/tecnologia-seguridad/operating-system/2023-06-19-03-instalacion-linux/index.pdf}{\faIcon{file-pdf}}
  \href{https://achalmaedison.netlify.app/tecnologia-seguridad/operating-system/2023-06-19-03-instalacion-linux}{03
  Instalacion Linux}
\item
  \href{https://achalmaedison.netlify.app/tecnologia-seguridad/operating-system/2023-06-20-04-administracion-particiones-volumenes/index.pdf}{\faIcon{file-pdf}}
  \href{https://achalmaedison.netlify.app/tecnologia-seguridad/operating-system/2023-06-20-04-administracion-particiones-volumenes}{04
  Administracion Particiones Volumenes}
\item
  \href{https://achalmaedison.netlify.app/tecnologia-seguridad/operating-system/2023-07-01-atajos-de-teclado-y-comandos-para-usar-vim/index.pdf}{\faIcon{file-pdf}}
  \href{https://achalmaedison.netlify.app/tecnologia-seguridad/operating-system/2023-07-01-atajos-de-teclado-y-comandos-para-usar-vim}{Atajos
  De Teclado Y Comandos Para Usar Vim}
\item
  \href{https://achalmaedison.netlify.app/tecnologia-seguridad/operating-system/2024-07-15-instalando-specitify/index.pdf}{\faIcon{file-pdf}}
  \href{https://achalmaedison.netlify.app/tecnologia-seguridad/operating-system/2024-07-15-instalando-specitify}{Instalando
  Specitify}
\end{enumerate}

Esperamos que encuentres estas publicaciones igualmente interesantes y
útiles. ¡Disfruta de la lectura!






\end{document}
