\documentclass[
  jou,
  floatsintext,
  longtable,
  a4paper,
  nolmodern,
  notxfonts,
  notimes,
  colorlinks=true,linkcolor=blue,citecolor=blue,urlcolor=blue]{apa7}

\usepackage{amsmath}
\usepackage{amssymb}



\usepackage[bidi=default]{babel}
\babelprovide[main,import]{spanish}


\babelfont{rm}[,RawFeature={fallback=mainfontfallback}]{Latin Modern
Roman}
% get rid of language-specific shorthands (see #6817):
\let\LanguageShortHands\languageshorthands
\def\languageshorthands#1{}

\RequirePackage{longtable}
\RequirePackage{threeparttablex}

\makeatletter
\renewcommand{\paragraph}{\@startsection{paragraph}{4}{\parindent}%
	{0\baselineskip \@plus 0.2ex \@minus 0.2ex}%
	{-.5em}%
	{\normalfont\normalsize\bfseries\typesectitle}}

\renewcommand{\subparagraph}[1]{\@startsection{subparagraph}{5}{0.5em}%
	{0\baselineskip \@plus 0.2ex \@minus 0.2ex}%
	{-\z@\relax}%
	{\normalfont\normalsize\bfseries\itshape\hspace{\parindent}{#1}\textit{\addperi}}{\relax}}
\makeatother




\usepackage{longtable, booktabs, multirow, multicol, colortbl, hhline, caption, array, float, xpatch}
\usepackage{subcaption}


\renewcommand\thesubfigure{\Alph{subfigure}}
\setcounter{topnumber}{2}
\setcounter{bottomnumber}{2}
\setcounter{totalnumber}{4}
\renewcommand{\topfraction}{0.85}
\renewcommand{\bottomfraction}{0.85}
\renewcommand{\textfraction}{0.15}
\renewcommand{\floatpagefraction}{0.7}

\usepackage{tcolorbox}
\tcbuselibrary{listings,theorems, breakable, skins}
\usepackage{fontawesome5}

\definecolor{quarto-callout-color}{HTML}{909090}
\definecolor{quarto-callout-note-color}{HTML}{0758E5}
\definecolor{quarto-callout-important-color}{HTML}{CC1914}
\definecolor{quarto-callout-warning-color}{HTML}{EB9113}
\definecolor{quarto-callout-tip-color}{HTML}{00A047}
\definecolor{quarto-callout-caution-color}{HTML}{FC5300}
\definecolor{quarto-callout-color-frame}{HTML}{ACACAC}
\definecolor{quarto-callout-note-color-frame}{HTML}{4582EC}
\definecolor{quarto-callout-important-color-frame}{HTML}{D9534F}
\definecolor{quarto-callout-warning-color-frame}{HTML}{F0AD4E}
\definecolor{quarto-callout-tip-color-frame}{HTML}{02B875}
\definecolor{quarto-callout-caution-color-frame}{HTML}{FD7E14}

%\newlength\Oldarrayrulewidth
%\newlength\Oldtabcolsep


\usepackage{hyperref}



\usepackage{color}
\usepackage{fancyvrb}
\newcommand{\VerbBar}{|}
\newcommand{\VERB}{\Verb[commandchars=\\\{\}]}
\DefineVerbatimEnvironment{Highlighting}{Verbatim}{commandchars=\\\{\}}
% Add ',fontsize=\small' for more characters per line
\usepackage{framed}
\definecolor{shadecolor}{RGB}{241,243,245}
\newenvironment{Shaded}{\begin{snugshade}}{\end{snugshade}}
\newcommand{\AlertTok}[1]{\textcolor[rgb]{0.68,0.00,0.00}{#1}}
\newcommand{\AnnotationTok}[1]{\textcolor[rgb]{0.37,0.37,0.37}{#1}}
\newcommand{\AttributeTok}[1]{\textcolor[rgb]{0.40,0.45,0.13}{#1}}
\newcommand{\BaseNTok}[1]{\textcolor[rgb]{0.68,0.00,0.00}{#1}}
\newcommand{\BuiltInTok}[1]{\textcolor[rgb]{0.00,0.23,0.31}{#1}}
\newcommand{\CharTok}[1]{\textcolor[rgb]{0.13,0.47,0.30}{#1}}
\newcommand{\CommentTok}[1]{\textcolor[rgb]{0.37,0.37,0.37}{#1}}
\newcommand{\CommentVarTok}[1]{\textcolor[rgb]{0.37,0.37,0.37}{\textit{#1}}}
\newcommand{\ConstantTok}[1]{\textcolor[rgb]{0.56,0.35,0.01}{#1}}
\newcommand{\ControlFlowTok}[1]{\textcolor[rgb]{0.00,0.23,0.31}{\textbf{#1}}}
\newcommand{\DataTypeTok}[1]{\textcolor[rgb]{0.68,0.00,0.00}{#1}}
\newcommand{\DecValTok}[1]{\textcolor[rgb]{0.68,0.00,0.00}{#1}}
\newcommand{\DocumentationTok}[1]{\textcolor[rgb]{0.37,0.37,0.37}{\textit{#1}}}
\newcommand{\ErrorTok}[1]{\textcolor[rgb]{0.68,0.00,0.00}{#1}}
\newcommand{\ExtensionTok}[1]{\textcolor[rgb]{0.00,0.23,0.31}{#1}}
\newcommand{\FloatTok}[1]{\textcolor[rgb]{0.68,0.00,0.00}{#1}}
\newcommand{\FunctionTok}[1]{\textcolor[rgb]{0.28,0.35,0.67}{#1}}
\newcommand{\ImportTok}[1]{\textcolor[rgb]{0.00,0.46,0.62}{#1}}
\newcommand{\InformationTok}[1]{\textcolor[rgb]{0.37,0.37,0.37}{#1}}
\newcommand{\KeywordTok}[1]{\textcolor[rgb]{0.00,0.23,0.31}{\textbf{#1}}}
\newcommand{\NormalTok}[1]{\textcolor[rgb]{0.00,0.23,0.31}{#1}}
\newcommand{\OperatorTok}[1]{\textcolor[rgb]{0.37,0.37,0.37}{#1}}
\newcommand{\OtherTok}[1]{\textcolor[rgb]{0.00,0.23,0.31}{#1}}
\newcommand{\PreprocessorTok}[1]{\textcolor[rgb]{0.68,0.00,0.00}{#1}}
\newcommand{\RegionMarkerTok}[1]{\textcolor[rgb]{0.00,0.23,0.31}{#1}}
\newcommand{\SpecialCharTok}[1]{\textcolor[rgb]{0.37,0.37,0.37}{#1}}
\newcommand{\SpecialStringTok}[1]{\textcolor[rgb]{0.13,0.47,0.30}{#1}}
\newcommand{\StringTok}[1]{\textcolor[rgb]{0.13,0.47,0.30}{#1}}
\newcommand{\VariableTok}[1]{\textcolor[rgb]{0.07,0.07,0.07}{#1}}
\newcommand{\VerbatimStringTok}[1]{\textcolor[rgb]{0.13,0.47,0.30}{#1}}
\newcommand{\WarningTok}[1]{\textcolor[rgb]{0.37,0.37,0.37}{\textit{#1}}}

\providecommand{\tightlist}{%
  \setlength{\itemsep}{0pt}\setlength{\parskip}{0pt}}
\usepackage{longtable,booktabs,array}
\usepackage{calc} % for calculating minipage widths
% Correct order of tables after \paragraph or \subparagraph
\usepackage{etoolbox}
\makeatletter
\patchcmd\longtable{\par}{\if@noskipsec\mbox{}\fi\par}{}{}
\makeatother
% Allow footnotes in longtable head/foot
\IfFileExists{footnotehyper.sty}{\usepackage{footnotehyper}}{\usepackage{footnote}}
\makesavenoteenv{longtable}

\usepackage{graphicx}
\makeatletter
\newsavebox\pandoc@box
\newcommand*\pandocbounded[1]{% scales image to fit in text height/width
  \sbox\pandoc@box{#1}%
  \Gscale@div\@tempa{\textheight}{\dimexpr\ht\pandoc@box+\dp\pandoc@box\relax}%
  \Gscale@div\@tempb{\linewidth}{\wd\pandoc@box}%
  \ifdim\@tempb\p@<\@tempa\p@\let\@tempa\@tempb\fi% select the smaller of both
  \ifdim\@tempa\p@<\p@\scalebox{\@tempa}{\usebox\pandoc@box}%
  \else\usebox{\pandoc@box}%
  \fi%
}
% Set default figure placement to htbp
\def\fps@figure{htbp}
\makeatother







\usepackage{fontspec} 

\defaultfontfeatures{Scale=MatchLowercase}
\defaultfontfeatures[\rmfamily]{Ligatures=TeX,Scale=1}

  \setmainfont[,RawFeature={fallback=mainfontfallback}]{Latin Modern
Roman}




\title{Guía de Git Cómo trabajar en equipo en proyectos: Aprende a usar
Git para controlar versiones, colaborar con otros desarrolladores y
mantener tu código organizado.}


\shorttitle{Editar}


\usepackage{etoolbox}



\ccoppy{\textcopyright~2021}



\author{Edison Achalma}



\affiliation{
{Escuela Profesional de Economía, Universidad Nacional de San Cristóbal
de Huamanga}}




\leftheader{Achalma}

\date{2021-08-17}


\abstract{Primer parrafo de abstrac }

\keywords{keyword1, keyword2}

\authornote{\par{\addORCIDlink{Edison Achalma}{0000-0001-6996-3364}} 
\par{ }
\par{   El autor no tiene conflictos de interés que revelar.    Los
roles de autor se clasificaron utilizando la taxonomía de roles de
colaborador (CRediT; https://credit.niso.org/) de la siguiente
manera:  Edison Achalma:   conceptualización, redacción}
\par{La correspondencia relativa a este artículo debe dirigirse a Edison
Achalma, Email: \href{mailto:elmer.achalma.09@unsch.edu.pe}{elmer.achalma.09@unsch.edu.pe}}
}

\usepackage{pbalance}
% \usepackage{float}
\makeatletter
\let\oldtpt\ThreePartTable
\let\endoldtpt\endThreePartTable
\def\ThreePartTable{\@ifnextchar[\ThreePartTable@i \ThreePartTable@ii}
\def\ThreePartTable@i[#1]{\begin{figure}[!htbp]
\onecolumn
\begin{minipage}{0.485\textwidth}
\oldtpt[#1]
}
\def\ThreePartTable@ii{\begin{figure}[!htbp]
\onecolumn
\begin{minipage}{0.48\textwidth}
\oldtpt
}
\def\endThreePartTable{
\endoldtpt
\end{minipage}
\twocolumn
\end{figure}}
\makeatother


\makeatletter
\let\endoldlt\endlongtable		
\def\endlongtable{
\hline
\endoldlt}
\makeatother

\newenvironment{twocolumntable}% environment name
{% begin code
\begin{table*}[!htbp]%
\onecolumn%
}%
{%
\twocolumn%
\end{table*}%
}% end code

\urlstyle{same}



\makeatletter
\@ifpackageloaded{caption}{}{\usepackage{caption}}
\AtBeginDocument{%
\ifdefined\contentsname
  \renewcommand*\contentsname{Tabla de contenidos}
\else
  \newcommand\contentsname{Tabla de contenidos}
\fi
\ifdefined\listfigurename
  \renewcommand*\listfigurename{List of Figures}
\else
  \newcommand\listfigurename{List of Figures}
\fi
\ifdefined\listtablename
  \renewcommand*\listtablename{List of Tables}
\else
  \newcommand\listtablename{List of Tables}
\fi
\ifdefined\figurename
  \renewcommand*\figurename{Figura}
\else
  \newcommand\figurename{Figura}
\fi
\ifdefined\tablename
  \renewcommand*\tablename{Tabla}
\else
  \newcommand\tablename{Tabla}
\fi
}
\@ifpackageloaded{float}{}{\usepackage{float}}
\floatstyle{ruled}
\@ifundefined{c@chapter}{\newfloat{codelisting}{h}{lop}}{\newfloat{codelisting}{h}{lop}[chapter]}
\floatname{codelisting}{Listing}
\newcommand*\listoflistings{\listof{codelisting}{List of Listings}}
\makeatother
\makeatletter
\makeatother
\makeatletter
\@ifpackageloaded{caption}{}{\usepackage{caption}}
\@ifpackageloaded{subcaption}{}{\usepackage{subcaption}}
\makeatother
\makeatletter
\@ifpackageloaded{fontawesome5}{}{\usepackage{fontawesome5}}
\makeatother

% From https://tex.stackexchange.com/a/645996/211326
%%% apa7 doesn't want to add appendix section titles in the toc
%%% let's make it do it
\makeatletter
\xpatchcmd{\appendix}
  {\par}
  {\addcontentsline{toc}{section}{\@currentlabelname}\par}
  {}{}
\makeatother

%% Disable longtable counter
%% https://tex.stackexchange.com/a/248395/211326

\usepackage{etoolbox}

\makeatletter
\patchcmd{\LT@caption}
  {\bgroup}
  {\bgroup\global\LTpatch@captiontrue}
  {}{}
\patchcmd{\longtable}
  {\par}
  {\par\global\LTpatch@captionfalse}
  {}{}
\apptocmd{\endlongtable}
  {\ifLTpatch@caption\else\addtocounter{table}{-1}\fi}
  {}{}
\newif\ifLTpatch@caption
\makeatother

\begin{document}

\maketitle

\hypertarget{toc}{}
\tableofcontents
\newpage
\section[Introduction]{Guía de Git Cómo trabajar en equipo en proyectos}

\setcounter{secnumdepth}{5}

\setlength\LTleft{0pt}


\section{Limpieza y Optimización de PC - Windows
11}\label{limpieza-y-optimizaciuxf3n-de-pc---windows-11}

\textbf{Pasos para Limpiar tu PC:}

\begin{enumerate}
\def\labelenumi{\arabic{enumi}.}
\item
  \textbf{Eliminación de Archivos Temporales:}

  \begin{itemize}
  \tightlist
  \item
    Presiona ``Win + R'' para abrir el cuadro de ejecución.
  \item
    Escribe \texttt{\%temp\%} y presiona Enter.
  \item
    Selecciona todos los archivos y elimínalos.
  \item
    Ejecuta nuevamente y escribe ``temp'' sin porcentajes.
  \item
    Elimina todos los archivos nuevamente.
  \end{itemize}
\item
  \textbf{Limpieza de Prefetch:}

  \begin{itemize}
  \tightlist
  \item
    Presiona ``Win + R''.
  \item
    Escribe ``prefetch'' y presiona Enter.
  \item
    Selecciona y elimina los archivos.
  \end{itemize}
\item
  \textbf{Limpieza de Registros:}

  \begin{itemize}
  \tightlist
  \item
    Presiona ``Win + R''.
  \item
    Escribe ``eventvwr'' y presiona Enter.
  \item
    En la ventana que se abre, ve a ``Windows Log'', luego ``Registro de
    Windows''.
  \item
    Vacía los registros de ``Aplicación'' y ``Eventos''.
  \end{itemize}
\item
  \textbf{Limpieza con Cleanmgr:}

  \begin{itemize}
  \item
    Ejecuta el siguiente comando en ``Ejecutar'':

\begin{Shaded}
\begin{Highlighting}[]
\ExtensionTok{\%SystemRoot\%\textbackslash{}System32\textbackslash{}Cmd.exe}\NormalTok{ /c Cleanmgr /sageset:65535 }\KeywordTok{\&}\ExtensionTok{Cleanmgr}\NormalTok{ /sagerun:65535}
\end{Highlighting}
\end{Shaded}
  \end{itemize}
\end{enumerate}

\textbf{Limpieza y Optimización de PC - Linux:}

\begin{enumerate}
\def\labelenumi{\arabic{enumi}.}
\item
  \textbf{Eliminación de Paquetes de Instalación:}

  \begin{itemize}
  \item
    Ejecuta el comando:

\begin{Shaded}
\begin{Highlighting}[]
\FunctionTok{sudo}\NormalTok{ apt{-}get autoremove}
\end{Highlighting}
\end{Shaded}
  \end{itemize}
\item
  \textbf{Limpieza de Paquetes en Caché:}

  \begin{itemize}
  \item
    Ejecuta los siguientes comandos:

\begin{Shaded}
\begin{Highlighting}[]
\FunctionTok{sudo}\NormalTok{ apt{-}get autoclean}
\FunctionTok{sudo}\NormalTok{ apt{-}get clean}
\end{Highlighting}
\end{Shaded}
  \end{itemize}
\item
  \textbf{Eliminación de Paquetes Huérfanos:}

  \begin{itemize}
  \item
    Utiliza el siguiente comando para verificar paquetes huérfanos:
  \item
\begin{Shaded}
\begin{Highlighting}[]
\FunctionTok{sudo}\NormalTok{ deborphan}
\end{Highlighting}
\end{Shaded}
  \end{itemize}
\end{enumerate}

\textbf{Recuerda que estos pasos pueden liberar espacio y mejorar el
rendimiento de tu sistema. Asegúrate de respaldar cualquier dato
importante antes de realizar estas operaciones.}

Este procedimiento te ayudará a mantener tu sistema limpio y eficiente.
No olvides revisar regularmente para mantener un rendimiento óptimo.

\section{Publicaciones Similares}\label{publicaciones-similares}

Si te interesó este artículo, te recomendamos que explores otros blogs y
recursos relacionados que pueden ampliar tus conocimientos. Aquí te dejo
algunas sugerencias:

\begin{enumerate}
\def\labelenumi{\arabic{enumi}.}
\tightlist
\item
  \href{https://achalmaedison.netlify.app/tecnologia-seguridad/operating-system/2017-05-21-comandos-de-informacion-windows/index.pdf}{\faIcon{file-pdf}}
  \href{https://achalmaedison.netlify.app/tecnologia-seguridad/operating-system/2017-05-21-comandos-de-informacion-windows}{Comandos
  De Informacion Windows}
\item
  \href{https://achalmaedison.netlify.app/tecnologia-seguridad/operating-system/2019-06-19-adb/index.pdf}{\faIcon{file-pdf}}
  \href{https://achalmaedison.netlify.app/tecnologia-seguridad/operating-system/2019-06-19-adb}{Adb}
\item
  \href{https://achalmaedison.netlify.app/tecnologia-seguridad/operating-system/2021-08-17-limpieza-y-optimizacion-de-pc/index.pdf}{\faIcon{file-pdf}}
  \href{https://achalmaedison.netlify.app/tecnologia-seguridad/operating-system/2021-08-17-limpieza-y-optimizacion-de-pc}{Limpieza
  Y Optimizacion De Pc}
\item
  \href{https://achalmaedison.netlify.app/tecnologia-seguridad/operating-system/2021-10-21-usando-apk-en-windown-11/index.pdf}{\faIcon{file-pdf}}
  \href{https://achalmaedison.netlify.app/tecnologia-seguridad/operating-system/2021-10-21-usando-apk-en-windown-11}{Usando
  Apk En Windown 11}
\item
  \href{https://achalmaedison.netlify.app/tecnologia-seguridad/operating-system/2022-05-12-gestionar-versiones-de-jdk-en-kubuntu/index.pdf}{\faIcon{file-pdf}}
  \href{https://achalmaedison.netlify.app/tecnologia-seguridad/operating-system/2022-05-12-gestionar-versiones-de-jdk-en-kubuntu}{Gestionar
  Versiones De Jdk En Kubuntu}
\item
  \href{https://achalmaedison.netlify.app/tecnologia-seguridad/operating-system/2022-07-21-instalar-tor-browser/index.pdf}{\faIcon{file-pdf}}
  \href{https://achalmaedison.netlify.app/tecnologia-seguridad/operating-system/2022-07-21-instalar-tor-browser}{Instalar
  Tor Browser}
\item
  \href{https://achalmaedison.netlify.app/tecnologia-seguridad/operating-system/2022-08-14-crear-enlaces-duros-o-hard-link-en-linux/index.pdf}{\faIcon{file-pdf}}
  \href{https://achalmaedison.netlify.app/tecnologia-seguridad/operating-system/2022-08-14-crear-enlaces-duros-o-hard-link-en-linux}{Crear
  Enlaces Duros O Hard Link En Linux}
\item
  \href{https://achalmaedison.netlify.app/tecnologia-seguridad/operating-system/2022-09-27-comandos-vim/index.pdf}{\faIcon{file-pdf}}
  \href{https://achalmaedison.netlify.app/tecnologia-seguridad/operating-system/2022-09-27-comandos-vim}{Comandos
  Vim}
\item
  \href{https://achalmaedison.netlify.app/tecnologia-seguridad/operating-system/2023-02-16-guia-de-git-y-github/index.pdf}{\faIcon{file-pdf}}
  \href{https://achalmaedison.netlify.app/tecnologia-seguridad/operating-system/2023-02-16-guia-de-git-y-github}{Guia
  De Git Y Github}
\item
  \href{https://achalmaedison.netlify.app/tecnologia-seguridad/operating-system/2023-05-02-00-primeros-pasos-en-linux/index.pdf}{\faIcon{file-pdf}}
  \href{https://achalmaedison.netlify.app/tecnologia-seguridad/operating-system/2023-05-02-00-primeros-pasos-en-linux}{00
  Primeros Pasos En Linux}
\item
  \href{https://achalmaedison.netlify.app/tecnologia-seguridad/operating-system/2023-06-17-01-introduccion-linux/index.pdf}{\faIcon{file-pdf}}
  \href{https://achalmaedison.netlify.app/tecnologia-seguridad/operating-system/2023-06-17-01-introduccion-linux}{01
  Introduccion Linux}
\item
  \href{https://achalmaedison.netlify.app/tecnologia-seguridad/operating-system/2023-06-18-02-distribuciones-linux/index.pdf}{\faIcon{file-pdf}}
  \href{https://achalmaedison.netlify.app/tecnologia-seguridad/operating-system/2023-06-18-02-distribuciones-linux}{02
  Distribuciones Linux}
\item
  \href{https://achalmaedison.netlify.app/tecnologia-seguridad/operating-system/2023-06-19-03-instalacion-linux/index.pdf}{\faIcon{file-pdf}}
  \href{https://achalmaedison.netlify.app/tecnologia-seguridad/operating-system/2023-06-19-03-instalacion-linux}{03
  Instalacion Linux}
\item
  \href{https://achalmaedison.netlify.app/tecnologia-seguridad/operating-system/2023-06-20-04-administracion-particiones-volumenes/index.pdf}{\faIcon{file-pdf}}
  \href{https://achalmaedison.netlify.app/tecnologia-seguridad/operating-system/2023-06-20-04-administracion-particiones-volumenes}{04
  Administracion Particiones Volumenes}
\item
  \href{https://achalmaedison.netlify.app/tecnologia-seguridad/operating-system/2023-07-01-atajos-de-teclado-y-comandos-para-usar-vim/index.pdf}{\faIcon{file-pdf}}
  \href{https://achalmaedison.netlify.app/tecnologia-seguridad/operating-system/2023-07-01-atajos-de-teclado-y-comandos-para-usar-vim}{Atajos
  De Teclado Y Comandos Para Usar Vim}
\item
  \href{https://achalmaedison.netlify.app/tecnologia-seguridad/operating-system/2024-07-15-instalando-specitify/index.pdf}{\faIcon{file-pdf}}
  \href{https://achalmaedison.netlify.app/tecnologia-seguridad/operating-system/2024-07-15-instalando-specitify}{Instalando
  Specitify}
\end{enumerate}

Esperamos que encuentres estas publicaciones igualmente interesantes y
útiles. ¡Disfruta de la lectura!






\end{document}
