\documentclass[
  jou,
  floatsintext,
  longtable,
  a4paper,
  nolmodern,
  notxfonts,
  notimes,
  colorlinks=true,linkcolor=blue,citecolor=blue,urlcolor=blue]{apa7}

\usepackage{amsmath}
\usepackage{amssymb}



\usepackage[bidi=default]{babel}
\babelprovide[main,import]{spanish}


\babelfont{rm}[,RawFeature={fallback=mainfontfallback}]{Latin Modern
Roman}
% get rid of language-specific shorthands (see #6817):
\let\LanguageShortHands\languageshorthands
\def\languageshorthands#1{}

\RequirePackage{longtable}
\RequirePackage{threeparttablex}

\makeatletter
\renewcommand{\paragraph}{\@startsection{paragraph}{4}{\parindent}%
	{0\baselineskip \@plus 0.2ex \@minus 0.2ex}%
	{-.5em}%
	{\normalfont\normalsize\bfseries\typesectitle}}

\renewcommand{\subparagraph}[1]{\@startsection{subparagraph}{5}{0.5em}%
	{0\baselineskip \@plus 0.2ex \@minus 0.2ex}%
	{-\z@\relax}%
	{\normalfont\normalsize\bfseries\itshape\hspace{\parindent}{#1}\textit{\addperi}}{\relax}}
\makeatother




\usepackage{longtable, booktabs, multirow, multicol, colortbl, hhline, caption, array, float, xpatch}
\usepackage{subcaption}


\renewcommand\thesubfigure{\Alph{subfigure}}
\setcounter{topnumber}{2}
\setcounter{bottomnumber}{2}
\setcounter{totalnumber}{4}
\renewcommand{\topfraction}{0.85}
\renewcommand{\bottomfraction}{0.85}
\renewcommand{\textfraction}{0.15}
\renewcommand{\floatpagefraction}{0.7}

\usepackage{tcolorbox}
\tcbuselibrary{listings,theorems, breakable, skins}
\usepackage{fontawesome5}

\definecolor{quarto-callout-color}{HTML}{909090}
\definecolor{quarto-callout-note-color}{HTML}{0758E5}
\definecolor{quarto-callout-important-color}{HTML}{CC1914}
\definecolor{quarto-callout-warning-color}{HTML}{EB9113}
\definecolor{quarto-callout-tip-color}{HTML}{00A047}
\definecolor{quarto-callout-caution-color}{HTML}{FC5300}
\definecolor{quarto-callout-color-frame}{HTML}{ACACAC}
\definecolor{quarto-callout-note-color-frame}{HTML}{4582EC}
\definecolor{quarto-callout-important-color-frame}{HTML}{D9534F}
\definecolor{quarto-callout-warning-color-frame}{HTML}{F0AD4E}
\definecolor{quarto-callout-tip-color-frame}{HTML}{02B875}
\definecolor{quarto-callout-caution-color-frame}{HTML}{FD7E14}

%\newlength\Oldarrayrulewidth
%\newlength\Oldtabcolsep


\usepackage{hyperref}



\usepackage{color}
\usepackage{fancyvrb}
\newcommand{\VerbBar}{|}
\newcommand{\VERB}{\Verb[commandchars=\\\{\}]}
\DefineVerbatimEnvironment{Highlighting}{Verbatim}{commandchars=\\\{\}}
% Add ',fontsize=\small' for more characters per line
\usepackage{framed}
\definecolor{shadecolor}{RGB}{241,243,245}
\newenvironment{Shaded}{\begin{snugshade}}{\end{snugshade}}
\newcommand{\AlertTok}[1]{\textcolor[rgb]{0.68,0.00,0.00}{#1}}
\newcommand{\AnnotationTok}[1]{\textcolor[rgb]{0.37,0.37,0.37}{#1}}
\newcommand{\AttributeTok}[1]{\textcolor[rgb]{0.40,0.45,0.13}{#1}}
\newcommand{\BaseNTok}[1]{\textcolor[rgb]{0.68,0.00,0.00}{#1}}
\newcommand{\BuiltInTok}[1]{\textcolor[rgb]{0.00,0.23,0.31}{#1}}
\newcommand{\CharTok}[1]{\textcolor[rgb]{0.13,0.47,0.30}{#1}}
\newcommand{\CommentTok}[1]{\textcolor[rgb]{0.37,0.37,0.37}{#1}}
\newcommand{\CommentVarTok}[1]{\textcolor[rgb]{0.37,0.37,0.37}{\textit{#1}}}
\newcommand{\ConstantTok}[1]{\textcolor[rgb]{0.56,0.35,0.01}{#1}}
\newcommand{\ControlFlowTok}[1]{\textcolor[rgb]{0.00,0.23,0.31}{\textbf{#1}}}
\newcommand{\DataTypeTok}[1]{\textcolor[rgb]{0.68,0.00,0.00}{#1}}
\newcommand{\DecValTok}[1]{\textcolor[rgb]{0.68,0.00,0.00}{#1}}
\newcommand{\DocumentationTok}[1]{\textcolor[rgb]{0.37,0.37,0.37}{\textit{#1}}}
\newcommand{\ErrorTok}[1]{\textcolor[rgb]{0.68,0.00,0.00}{#1}}
\newcommand{\ExtensionTok}[1]{\textcolor[rgb]{0.00,0.23,0.31}{#1}}
\newcommand{\FloatTok}[1]{\textcolor[rgb]{0.68,0.00,0.00}{#1}}
\newcommand{\FunctionTok}[1]{\textcolor[rgb]{0.28,0.35,0.67}{#1}}
\newcommand{\ImportTok}[1]{\textcolor[rgb]{0.00,0.46,0.62}{#1}}
\newcommand{\InformationTok}[1]{\textcolor[rgb]{0.37,0.37,0.37}{#1}}
\newcommand{\KeywordTok}[1]{\textcolor[rgb]{0.00,0.23,0.31}{\textbf{#1}}}
\newcommand{\NormalTok}[1]{\textcolor[rgb]{0.00,0.23,0.31}{#1}}
\newcommand{\OperatorTok}[1]{\textcolor[rgb]{0.37,0.37,0.37}{#1}}
\newcommand{\OtherTok}[1]{\textcolor[rgb]{0.00,0.23,0.31}{#1}}
\newcommand{\PreprocessorTok}[1]{\textcolor[rgb]{0.68,0.00,0.00}{#1}}
\newcommand{\RegionMarkerTok}[1]{\textcolor[rgb]{0.00,0.23,0.31}{#1}}
\newcommand{\SpecialCharTok}[1]{\textcolor[rgb]{0.37,0.37,0.37}{#1}}
\newcommand{\SpecialStringTok}[1]{\textcolor[rgb]{0.13,0.47,0.30}{#1}}
\newcommand{\StringTok}[1]{\textcolor[rgb]{0.13,0.47,0.30}{#1}}
\newcommand{\VariableTok}[1]{\textcolor[rgb]{0.07,0.07,0.07}{#1}}
\newcommand{\VerbatimStringTok}[1]{\textcolor[rgb]{0.13,0.47,0.30}{#1}}
\newcommand{\WarningTok}[1]{\textcolor[rgb]{0.37,0.37,0.37}{\textit{#1}}}

\providecommand{\tightlist}{%
  \setlength{\itemsep}{0pt}\setlength{\parskip}{0pt}}
\usepackage{longtable,booktabs,array}
\usepackage{calc} % for calculating minipage widths
% Correct order of tables after \paragraph or \subparagraph
\usepackage{etoolbox}
\makeatletter
\patchcmd\longtable{\par}{\if@noskipsec\mbox{}\fi\par}{}{}
\makeatother
% Allow footnotes in longtable head/foot
\IfFileExists{footnotehyper.sty}{\usepackage{footnotehyper}}{\usepackage{footnote}}
\makesavenoteenv{longtable}

\usepackage{graphicx}
\makeatletter
\newsavebox\pandoc@box
\newcommand*\pandocbounded[1]{% scales image to fit in text height/width
  \sbox\pandoc@box{#1}%
  \Gscale@div\@tempa{\textheight}{\dimexpr\ht\pandoc@box+\dp\pandoc@box\relax}%
  \Gscale@div\@tempb{\linewidth}{\wd\pandoc@box}%
  \ifdim\@tempb\p@<\@tempa\p@\let\@tempa\@tempb\fi% select the smaller of both
  \ifdim\@tempa\p@<\p@\scalebox{\@tempa}{\usebox\pandoc@box}%
  \else\usebox{\pandoc@box}%
  \fi%
}
% Set default figure placement to htbp
\def\fps@figure{htbp}
\makeatother







\usepackage{fontspec} 

\defaultfontfeatures{Scale=MatchLowercase}
\defaultfontfeatures[\rmfamily]{Ligatures=TeX,Scale=1}

  \setmainfont[,RawFeature={fallback=mainfontfallback}]{Latin Modern
Roman}




\title{Cómo Gestionar Dotfiles con GNU Stow: Guía Práctica}


\shorttitle{Gestión de Dotfiles con GNU Stow}


\usepackage{etoolbox}



\ccoppy{\textcopyright~2025}



\author{Edison Achalma}



\affiliation{
{Escuela Profesional de Economía, Universidad Nacional de San Cristóbal
de Huamanga}}




\leftheader{Achalma}

\date{2025-07-10}


\abstract{This tutorial provides a step-by-step guide to managing
dotfiles using GNU Stow, a tool that leverages symbolic links to
centralize and synchronize configuration files across Unix-like systems
(Linux, macOS, WSL). It explains the importance of dotfiles, such as
.bashrc and .gitconfig, in customizing user environments and highlights
the inefficiencies of manual management. The guide details installing
GNU Stow, creating a dotfiles repository, linking configurations, and
automating the process with a bash script. Advanced tips include
handling conflicts, platform-specific setups, and alternatives like
Chezmoi and YADM. This resource is designed for developers seeking
efficient, portable configuration management. }

\keywords{Dotfiles, GNU Stow, Symbolic links, Configuration
management, Git integration}

\authornote{\par{\addORCIDlink{Edison Achalma}{0000-0001-6996-3364}} 
\par{ }
\par{   El autor no tiene conflictos de interés que revelar.    Los
roles de autor se clasificaron utilizando la taxonomía de roles de
colaborador (CRediT; https://credit.niso.org/) de la siguiente
manera:  Edison Achalma:   conceptualización, redacción}
\par{La correspondencia relativa a este artículo debe dirigirse a Edison
Achalma, Email: \href{mailto:elmer.achalma.09@unsch.edu.pe}{elmer.achalma.09@unsch.edu.pe}}
}

\usepackage{pbalance}
% \usepackage{float}
\makeatletter
\let\oldtpt\ThreePartTable
\let\endoldtpt\endThreePartTable
\def\ThreePartTable{\@ifnextchar[\ThreePartTable@i \ThreePartTable@ii}
\def\ThreePartTable@i[#1]{\begin{figure}[!htbp]
\onecolumn
\begin{minipage}{0.485\textwidth}
\oldtpt[#1]
}
\def\ThreePartTable@ii{\begin{figure}[!htbp]
\onecolumn
\begin{minipage}{0.48\textwidth}
\oldtpt
}
\def\endThreePartTable{
\endoldtpt
\end{minipage}
\twocolumn
\end{figure}}
\makeatother


\makeatletter
\let\endoldlt\endlongtable		
\def\endlongtable{
\hline
\endoldlt}
\makeatother

\newenvironment{twocolumntable}% environment name
{% begin code
\begin{table*}[!htbp]%
\onecolumn%
}%
{%
\twocolumn%
\end{table*}%
}% end code

\urlstyle{same}



\makeatletter
\@ifpackageloaded{caption}{}{\usepackage{caption}}
\AtBeginDocument{%
\ifdefined\contentsname
  \renewcommand*\contentsname{Tabla de contenidos}
\else
  \newcommand\contentsname{Tabla de contenidos}
\fi
\ifdefined\listfigurename
  \renewcommand*\listfigurename{List of Figures}
\else
  \newcommand\listfigurename{List of Figures}
\fi
\ifdefined\listtablename
  \renewcommand*\listtablename{List of Tables}
\else
  \newcommand\listtablename{List of Tables}
\fi
\ifdefined\figurename
  \renewcommand*\figurename{Figura}
\else
  \newcommand\figurename{Figura}
\fi
\ifdefined\tablename
  \renewcommand*\tablename{Tabla}
\else
  \newcommand\tablename{Tabla}
\fi
}
\@ifpackageloaded{float}{}{\usepackage{float}}
\floatstyle{ruled}
\@ifundefined{c@chapter}{\newfloat{codelisting}{h}{lop}}{\newfloat{codelisting}{h}{lop}[chapter]}
\floatname{codelisting}{Listing}
\newcommand*\listoflistings{\listof{codelisting}{List of Listings}}
\makeatother
\makeatletter
\makeatother
\makeatletter
\@ifpackageloaded{caption}{}{\usepackage{caption}}
\@ifpackageloaded{subcaption}{}{\usepackage{subcaption}}
\makeatother
\makeatletter
\@ifpackageloaded{fontawesome5}{}{\usepackage{fontawesome5}}
\makeatother

% From https://tex.stackexchange.com/a/645996/211326
%%% apa7 doesn't want to add appendix section titles in the toc
%%% let's make it do it
\makeatletter
\xpatchcmd{\appendix}
  {\par}
  {\addcontentsline{toc}{section}{\@currentlabelname}\par}
  {}{}
\makeatother

%% Disable longtable counter
%% https://tex.stackexchange.com/a/248395/211326

\usepackage{etoolbox}

\makeatletter
\patchcmd{\LT@caption}
  {\bgroup}
  {\bgroup\global\LTpatch@captiontrue}
  {}{}
\patchcmd{\longtable}
  {\par}
  {\par\global\LTpatch@captionfalse}
  {}{}
\apptocmd{\endlongtable}
  {\ifLTpatch@caption\else\addtocounter{table}{-1}\fi}
  {}{}
\newif\ifLTpatch@caption
\makeatother

\begin{document}

\maketitle

\hypertarget{toc}{}
\tableofcontents
\newpage
\section[Introduction]{Cómo Gestionar Dotfiles con GNU Stow}

\setcounter{secnumdepth}{5}

\setlength\LTleft{0pt}


¿Alguna vez has perdido horas configurando tu terminal o editor tras
cambiar de computadora? Los \textbf{dotfiles}, esos archivos ocultos
como \texttt{.bashrc} o \texttt{.gitconfig}, guardan tus
personalizaciones, pero gestionarlos a mano es un caos. \textbf{GNU
Stow} simplifica todo: organiza tus configuraciones en un repositorio
central y usa \textbf{enlaces simbólicos} para sincronizarlas en
minutos. Este tutorial te guía paso a paso para instalar Stow, crear un
repositorio de \textbf{dotfiles}, enlazar configuraciones y automatizar
el proceso. ¡Di adiós a las configuraciones repetitivas y toma el
control de tu entorno!

Con Stow, tus \textbf{configuraciones personalizadas} estarán siempre a
un comando de distancia. Aprenderás a centralizar archivos como
\texttt{.zshrc} o \texttt{.config/nvim}, integrarlos con Git y
desplegarlos en Linux, macOS o WSL sin complicaciones. ¿Listo para
optimizar tu flujo de trabajo? Sigue leyendo y descubre cómo \textbf{GNU
Stow} transforma la gestión de dotfiles en algo simple y poderoso.

\subsection{¿Qué son los Dotfiles y Por Qué
Importan?}\label{quuxe9-son-los-dotfiles-y-por-quuxe9-importan}

\subsubsection{Definición de Dotfiles y su
Rol}\label{definiciuxf3n-de-dotfiles-y-su-rol}

Los \textbf{dotfiles} son archivos ocultos en sistemas Unix-like (Linux,
macOS) que empiezan con un punto (ej., \texttt{.zshrc},
\texttt{.gitconfig}, \texttt{.config/nvim}). Almacenan configuraciones
personalizadas para tu terminal, editor de código o gestor de ventanas.
Por ejemplo, \texttt{.bashrc} define alias y variables de entorno,
mientras que \texttt{.vimrc} ajusta tu editor Vim. Estos archivos son el
corazón de tu flujo de trabajo, ya que personalizan tus herramientas
favoritas.

\subsubsection{Impacto en la Productividad del
Usuario}\label{impacto-en-la-productividad-del-usuario}

Tener \textbf{dotfiles} bien organizados te ahorra horas al replicar tu
entorno en nuevas máquinas. Imagina configurar tu shell o editor desde
cero tras reinstalar tu sistema: ¡es tedioso! Con una gestión adecuada,
puedes clonar tus configuraciones y tener todo listo rápidamente. Esto
es clave para desarrolladores que trabajan en múltiples dispositivos o
entornos como servidores y laptops.

\subsubsection{Problemas de la Gestión
Manual}\label{problemas-de-la-gestiuxf3n-manual}

Copiar \textbf{dotfiles} manualmente o usar scripts caseros es lento y
arriesgado. Puedes sobrescribir archivos, olvidar configuraciones o
perderlas en una reinstalación. Por ejemplo, mover \texttt{.zshrc} a
otra máquina sin un sistema organizado puede causar errores si las
versiones del software difieren. \textbf{GNU Stow} soluciona esto al
centralizar tus archivos y crear \textbf{enlaces simbólicos}
automáticamente, manteniendo todo sincronizado.

\subsection{¿Qué es GNU Stow y Cómo
Funciona?}\label{quuxe9-es-gnu-stow-y-cuxf3mo-funciona}

\subsubsection{Introducción a GNU Stow: Gestión de
Symlinks}\label{introducciuxf3n-a-gnu-stow-gestiuxf3n-de-symlinks}

\textbf{GNU Stow} es una herramienta que crea y gestiona \textbf{enlaces
simbólicos} (symlinks) para tus \textbf{dotfiles}. En lugar de copiar
archivos como \texttt{.bashrc} a tu directorio home
(\texttt{\textasciitilde{}}), Stow los mantiene en un repositorio
central (ej., \texttt{\textasciitilde{}/dotfiles}) y crea enlaces a las
ubicaciones correctas. Esto asegura que tus aplicaciones usen las
configuraciones sin duplicar archivos, y los cambios se reflejan en el
repositorio.

\subsubsection{Concepto de Paquetes en
Stow}\label{concepto-de-paquetes-en-stow}

Stow organiza tus \textbf{dotfiles} en \textbf{paquetes}, que son
subdirectorios en \texttt{\textasciitilde{}/dotfiles} (ej.,
\texttt{zsh}, \texttt{git}, \texttt{nvim}). Cada paquete replica la
estructura del sistema. Por ejemplo, para \texttt{.zshrc}, creas
\texttt{\textasciitilde{}/dotfiles/zsh/.zshrc}. Al ejecutar
\texttt{stow\ zsh}, Stow enlaza
\texttt{\textasciitilde{}/dotfiles/zsh/.zshrc} a
\texttt{\textasciitilde{}/.zshrc}. Esta modularidad te permite instalar
solo las configuraciones que necesitas en cada máquina.

\textbf{Ejemplo de Estructura de Repositorio de Dotfiles con Stow:}

Imagina que tu directorio principal de dotfiles se llama
\texttt{\textasciitilde{}/dotfiles/}. Dentro de él, tendrías
subdirectorios para cada ``paquete'':

\begin{verbatim}
~/dotfiles/
├── git/
│   └── .gitconfig
│   └── .gitignore_global
├── zsh/
│   └── .zshrc
│   └── .p10k.zsh
└── nvim/
│   └── .config/
│       └── nvim/
│           ├── init.lua
│           └── lua/
│               └── plugins.lua
│
├── .gitignore
└── install.sh
\end{verbatim}

\subsubsection{Beneficios de Usar Stow para
Dotfiles}\label{beneficios-de-usar-stow-para-dotfiles}

\begin{itemize}
\tightlist
\item
  \textbf{Centralización}: Todos tus \textbf{dotfiles} viven en un solo
  lugar, fáciles de versionar con Git.
\item
  \textbf{Modularidad}: Instala configuraciones específicas (ej., solo
  \texttt{git}) sin tocar otras.
\item
  \textbf{Sincronización}: Combina Stow con Git para clonar y desplegar
  configuraciones en cualquier sistema.
\item
  \textbf{Simplicidad}: Comandos como \texttt{stow\ zsh} hacen el
  trabajo pesado por ti.
\item
  \textbf{Portabilidad}: Funciona en Linux, macOS y WSL, ideal para
  entornos mixtos.
\end{itemize}

\subsection{Guía Práctica: Configura tus Dotfiles con
Stow}\label{guuxeda-pruxe1ctica-configura-tus-dotfiles-con-stow}

\subsubsection{Paso 1: Instala GNU Stow en tu
Sistema}\label{paso-1-instala-gnu-stow-en-tu-sistema}

Primero, instala \textbf{GNU Stow} en tu sistema. Usa el gestor de
paquetes de tu distribución:

\begin{itemize}
\item
  \textbf{Debian/Ubuntu}:

\begin{Shaded}
\begin{Highlighting}[]
\FunctionTok{sudo}\NormalTok{ apt update}
\FunctionTok{sudo}\NormalTok{ apt install stow}
\end{Highlighting}
\end{Shaded}
\item
  \textbf{Fedora}:

\begin{Shaded}
\begin{Highlighting}[]
\FunctionTok{sudo}\NormalTok{ dnf install stow}
\end{Highlighting}
\end{Shaded}
\item
  \textbf{Arch Linux}:

\begin{Shaded}
\begin{Highlighting}[]
\FunctionTok{sudo}\NormalTok{ pacman }\AttributeTok{{-}S}\NormalTok{ stow}
\end{Highlighting}
\end{Shaded}
\item
  \textbf{macOS (con Homebrew)}:

\begin{Shaded}
\begin{Highlighting}[]
\ExtensionTok{brew}\NormalTok{ install stow}
\end{Highlighting}
\end{Shaded}
\item
  \textbf{Windows (WSL)}: Usa los comandos de Ubuntu dentro de WSL.
\end{itemize}

Verifica la instalación:

\begin{Shaded}
\begin{Highlighting}[]
\ExtensionTok{stow} \AttributeTok{{-}{-}version}
\end{Highlighting}
\end{Shaded}

Si ves la versión (ej., \texttt{stow\ 2.3.1}), estás listo.

\subsubsection{Paso 2: Crea y Organiza tu Repositorio de
Dotfiles}\label{paso-2-crea-y-organiza-tu-repositorio-de-dotfiles}

\begin{enumerate}
\def\labelenumi{\arabic{enumi}.}
\item
  \textbf{Crea el directorio de dotfiles}:

\begin{Shaded}
\begin{Highlighting}[]
\FunctionTok{mkdir}\NormalTok{ \textasciitilde{}/dotfiles}
\BuiltInTok{cd}\NormalTok{ \textasciitilde{}/dotfiles}
\end{Highlighting}
\end{Shaded}
\item
  \textbf{Inicializa un repositorio Git} (para versionado y
  sincronización):

\begin{Shaded}
\begin{Highlighting}[]
\FunctionTok{git}\NormalTok{ init}
\end{Highlighting}
\end{Shaded}
\item
  \textbf{Crea paquetes para tus configuraciones}. Por ejemplo, para
  \texttt{.gitconfig}, \texttt{.zshrc} y \texttt{.config/nvim}:

\begin{Shaded}
\begin{Highlighting}[]
\FunctionTok{mkdir} \AttributeTok{{-}p}\NormalTok{ git zsh nvim/.config/nvim}
\end{Highlighting}
\end{Shaded}
\item
  \textbf{Mueve tus dotfiles existentes a los paquetes}. Ejemplo:

\begin{Shaded}
\begin{Highlighting}[]
\FunctionTok{mv}\NormalTok{ \textasciitilde{}/.gitconfig \textasciitilde{}/dotfiles/git/}
\FunctionTok{mv}\NormalTok{ \textasciitilde{}/.zshrc \textasciitilde{}/dotfiles/zsh/}
\FunctionTok{mv}\NormalTok{ \textasciitilde{}/.config/nvim/}\PreprocessorTok{*}\NormalTok{ \textasciitilde{}/dotfiles/nvim/.config/nvim/}
\end{Highlighting}
\end{Shaded}
\item
  \textbf{Crea un \texttt{.gitignore}} para evitar subir archivos
  sensibles o temporales:

\begin{Shaded}
\begin{Highlighting}[]
\NormalTok{*.bak}
\NormalTok{*.swp}
\NormalTok{.DS\_Store}
\NormalTok{nvim/.local/share/nvim/}
\end{Highlighting}
\end{Shaded}
\item
  \textbf{Commitea los cambios}:

\begin{Shaded}
\begin{Highlighting}[]
\FunctionTok{git}\NormalTok{ add .}
\FunctionTok{git}\NormalTok{ commit }\AttributeTok{{-}m} \StringTok{"Inicializar dotfiles"}
\FunctionTok{git}\NormalTok{ remote add origin https://github.com/tu{-}usuario/dotfiles.git}
\FunctionTok{git}\NormalTok{ push }\AttributeTok{{-}u}\NormalTok{ origin main}
\end{Highlighting}
\end{Shaded}
\end{enumerate}

Tu repositorio ahora está organizado y listo para Stow.

\subsubsection{Paso 3: Usa Stow para Enlazar
Configuraciones}\label{paso-3-usa-stow-para-enlazar-configuraciones}

\begin{enumerate}
\def\labelenumi{\arabic{enumi}.}
\item
  \textbf{Navega a \texttt{\textasciitilde{}/dotfiles}}:

\begin{Shaded}
\begin{Highlighting}[]
\BuiltInTok{cd}\NormalTok{ \textasciitilde{}/dotfiles}
\end{Highlighting}
\end{Shaded}
\item
  \textbf{Enlaza un paquete específico}:

\begin{Shaded}
\begin{Highlighting}[]
\ExtensionTok{stow}\NormalTok{ git}
\end{Highlighting}
\end{Shaded}

  Esto crea un enlace simbólico:
  \texttt{\textasciitilde{}/.gitconfig\ -\textgreater{}\ \textasciitilde{}/dotfiles/git/.gitconfig}.
\item
  \textbf{Enlaza múltiples paquetes}:

\begin{Shaded}
\begin{Highlighting}[]
\ExtensionTok{stow}\NormalTok{ git zsh nvim}
\end{Highlighting}
\end{Shaded}
\item
  \textbf{Verifica los enlaces}:

\begin{Shaded}
\begin{Highlighting}[]
\FunctionTok{ls} \AttributeTok{{-}l}\NormalTok{ \textasciitilde{}/.gitconfig \textasciitilde{}/.zshrc \textasciitilde{}/.config/nvim}
\end{Highlighting}
\end{Shaded}

  Deberías ver algo como:

\begin{Shaded}
\begin{Highlighting}[]
\NormalTok{lrwxrwxrwx 1 usuario usuario 36 Jul 11 2025 /home/usuario/.gitconfig {-}\textgreater{} dotfiles/git/.gitconfig}
\NormalTok{lrwxrwxrwx 1 usuario usuario 34 Jul 11 2025 /home/usuario/.zshrc {-}\textgreater{} dotfiles/zsh/.zshrc}
\end{Highlighting}
\end{Shaded}
\item
  \textbf{Prueba en otra máquina}:

  \begin{itemize}
  \item
    Clona el repositorio:

\begin{Shaded}
\begin{Highlighting}[]
\FunctionTok{git}\NormalTok{ clone https://github.com/tu{-}usuario/dotfiles.git \textasciitilde{}/dotfiles}
\end{Highlighting}
\end{Shaded}
  \item
    Instala Stow y ejecuta:

\begin{Shaded}
\begin{Highlighting}[]
\BuiltInTok{cd}\NormalTok{ \textasciitilde{}/dotfiles}
\ExtensionTok{stow}\NormalTok{ git zsh nvim}
\end{Highlighting}
\end{Shaded}
  \end{itemize}
\end{enumerate}

\subsubsection{Paso 4: Automatiza con un Script de
Instalación}\label{paso-4-automatiza-con-un-script-de-instalaciuxf3n}

Crea un script \texttt{install.sh} para automatizar la instalación:

\begin{enumerate}
\def\labelenumi{\arabic{enumi}.}
\item
  \textbf{Crea el script}:

\begin{Shaded}
\begin{Highlighting}[]
\FunctionTok{nano}\NormalTok{ \textasciitilde{}/dotfiles/install.sh}
\end{Highlighting}
\end{Shaded}
\item
  \textbf{Añade este contenido}:

\begin{Shaded}
\begin{Highlighting}[]
\CommentTok{\#!/bin/bash}

\VariableTok{DOTFILES\_DIR}\OperatorTok{=}\StringTok{"}\VariableTok{$HOME}\StringTok{/dotfiles"}

\CommentTok{\# Verifica si Stow está instalado}
\ControlFlowTok{if} \OtherTok{! }\BuiltInTok{command} \AttributeTok{{-}v}\NormalTok{ stow }\OperatorTok{\&\textgreater{}}\NormalTok{ /dev/null}\KeywordTok{;} \ControlFlowTok{then}
    \BuiltInTok{echo} \StringTok{"Error: GNU Stow no está instalado. Instálalo con: sudo apt install stow"}
    \BuiltInTok{exit}\NormalTok{ 1}
\ControlFlowTok{fi}

\CommentTok{\# Enlaza todos los paquetes}
\BuiltInTok{cd} \StringTok{"}\VariableTok{$DOTFILES\_DIR}\StringTok{"} \KeywordTok{||} \BuiltInTok{exit}
\ExtensionTok{stow} \AttributeTok{{-}v}\NormalTok{ git zsh nvim}
\BuiltInTok{echo} \StringTok{"Dotfiles instalados correctamente!"}
\end{Highlighting}
\end{Shaded}
\item
  \textbf{Hazlo ejecutable}:

\begin{Shaded}
\begin{Highlighting}[]
\FunctionTok{chmod}\NormalTok{ +x \textasciitilde{}/dotfiles/install.sh}
\end{Highlighting}
\end{Shaded}
\item
  \textbf{Ejecuta el script}:

\begin{Shaded}
\begin{Highlighting}[]
\ExtensionTok{./install.sh}
\end{Highlighting}
\end{Shaded}
\item
  \textbf{Commitea el script}:

\begin{Shaded}
\begin{Highlighting}[]
\FunctionTok{git}\NormalTok{ add install.sh}
\FunctionTok{git}\NormalTok{ commit }\AttributeTok{{-}m} \StringTok{"Añadir script de instalación"}
\FunctionTok{git}\NormalTok{ push}
\end{Highlighting}
\end{Shaded}
\end{enumerate}

Este script simplifica el despliegue en cualquier máquina.

\subsection{Consejos Avanzados para Optimizar
Stow}\label{consejos-avanzados-para-optimizar-stow}

\subsubsection{Manejo de Conflictos con
--adopt}\label{manejo-de-conflictos-con-adopt}

Si un archivo como \texttt{\textasciitilde{}/.zshrc} ya existe, Stow
mostrará un error. Usa \texttt{-\/-adopt} para mover el archivo al
repositorio y enlazarlo:

\begin{enumerate}
\def\labelenumi{\arabic{enumi}.}
\item
  \textbf{Ejecuta con \texttt{-\/-adopt}}:

\begin{Shaded}
\begin{Highlighting}[]
\BuiltInTok{cd}\NormalTok{ \textasciitilde{}/dotfiles}
\ExtensionTok{stow} \AttributeTok{{-}{-}adopt}\NormalTok{ zsh}
\end{Highlighting}
\end{Shaded}
\item
  \textbf{Commitea los cambios}:

\begin{Shaded}
\begin{Highlighting}[]
\FunctionTok{git}\NormalTok{ add zsh/.zshrc}
\FunctionTok{git}\NormalTok{ commit }\AttributeTok{{-}m} \StringTok{"Adoptar zshrc existente"}
\FunctionTok{git}\NormalTok{ push}
\end{Highlighting}
\end{Shaded}
\end{enumerate}

\textbf{Precaución}: Haz un respaldo antes (ej.,
\texttt{cp\ \textasciitilde{}/.zshrc\ \textasciitilde{}/.zshrc.bak}).

\subsubsection{Desvinculación de Paquetes con
-D}\label{desvinculaciuxf3n-de-paquetes-con--d}

Para eliminar enlaces simbólicos sin borrar los archivos en
\texttt{\textasciitilde{}/dotfiles}:

\begin{enumerate}
\def\labelenumi{\arabic{enumi}.}
\item
  \textbf{Desvincula un paquete}:

\begin{Shaded}
\begin{Highlighting}[]
\BuiltInTok{cd}\NormalTok{ \textasciitilde{}/dotfiles}
\ExtensionTok{stow} \AttributeTok{{-}D}\NormalTok{ zsh}
\end{Highlighting}
\end{Shaded}
\item
  \textbf{Verifica}:

\begin{Shaded}
\begin{Highlighting}[]
\FunctionTok{ls} \AttributeTok{{-}l}\NormalTok{ \textasciitilde{}/.zshrc}
\end{Highlighting}
\end{Shaded}

  El enlace debería haber desaparecido, pero
  \texttt{\textasciitilde{}/dotfiles/zsh/.zshrc} permanece.
\end{enumerate}

\subsubsection{Compatibilidad Multiplataforma y
Portabilidad}\label{compatibilidad-multiplataforma-y-portabilidad}

Stow funciona en Linux, macOS y WSL. Para configuraciones específicas:

\begin{enumerate}
\def\labelenumi{\arabic{enumi}.}
\item
  \textbf{Crea paquetes por sistema}. Ejemplo: \texttt{kde} para Linux,
  \texttt{zsh-macos} para macOS.
\item
  \textbf{Usa ramas en Git}:

\begin{Shaded}
\begin{Highlighting}[]
\FunctionTok{git}\NormalTok{ checkout }\AttributeTok{{-}b}\NormalTok{ macos}
\FunctionTok{git}\NormalTok{ add zsh{-}macos}
\FunctionTok{git}\NormalTok{ commit }\AttributeTok{{-}m} \StringTok{"Configuraciones para macOS"}
\FunctionTok{git}\NormalTok{ push origin macos}
\end{Highlighting}
\end{Shaded}
\item
  \textbf{Instala selectivamente}:

\begin{Shaded}
\begin{Highlighting}[]
\ExtensionTok{stow}\NormalTok{ zsh{-}macos}
\end{Highlighting}
\end{Shaded}
\end{enumerate}

Esto asegura que solo uses configuraciones relevantes por máquina.

\subsection{Alternativas a GNU Stow: ¿Qué Opciones
Existen?}\label{alternativas-a-gnu-stow-quuxe9-opciones-existen}

\subsubsection{Repositorios Git Bare: Simplicidad y
Riesgos}\label{repositorios-git-bare-simplicidad-y-riesgos}

Un repositorio Git ``bare'' usa \texttt{\$HOME} como área de trabajo:

\begin{Shaded}
\begin{Highlighting}[]
\FunctionTok{git}\NormalTok{ init }\AttributeTok{{-}{-}bare}\NormalTok{ \textasciitilde{}/.dotfiles}
\BuiltInTok{alias}\NormalTok{ config=}\StringTok{\textquotesingle{}/usr/bin/git {-}{-}git{-}dir=$HOME/.dotfiles/ {-}{-}work{-}tree=$HOME\textquotesingle{}}
\ExtensionTok{config}\NormalTok{ add .zshrc}
\ExtensionTok{config}\NormalTok{ commit }\AttributeTok{{-}m} \StringTok{"Añadir zshrc"}
\end{Highlighting}
\end{Shaded}

\textbf{Ventajas}: Simple, no requiere herramientas adicionales.
\textbf{Riesgos}: Puedes subir archivos sensibles si no configuras
\texttt{.gitignore}.

\subsubsection{Chezmoi y YADM: Herramientas
Modernas}\label{chezmoi-y-yadm-herramientas-modernas}

\begin{itemize}
\item
  \textbf{Chezmoi}: Gestiona dotfiles con plantillas y cifrado. Ideal
  para múltiples sistemas.

\begin{Shaded}
\begin{Highlighting}[]
\ExtensionTok{chezmoi}\NormalTok{ init}
\ExtensionTok{chezmoi}\NormalTok{ add \textasciitilde{}/.zshrc}
\end{Highlighting}
\end{Shaded}
\item
  \textbf{YADM}: Wrapper de Git con funciones como alternates.

\begin{Shaded}
\begin{Highlighting}[]
\ExtensionTok{yadm}\NormalTok{ init}
\ExtensionTok{yadm}\NormalTok{ add \textasciitilde{}/.zshrc}
\end{Highlighting}
\end{Shaded}

  \textbf{Ventajas}: Más funciones que Stow, como gestión de secretos.
  \textbf{Desventajas}: Mayor curva de aprendizaje.
\end{itemize}

\subsubsection{Home Manager: Configuración
Declarativa}\label{home-manager-configuraciuxf3n-declarativa}

\textbf{Home Manager} (para NixOS) define dotfiles y paquetes
declarativamente:

\begin{Shaded}
\begin{Highlighting}[]
\ExtensionTok{home{-}manager}\NormalTok{ switch}
\end{Highlighting}
\end{Shaded}

\textbf{Ventajas}: Configuración completa del sistema.
\textbf{Desventajas}: Complejo, requiere aprender Nix.

\subsection{Conclusión: Controla tu Entorno
Digital}\label{conclusiuxf3n-controla-tu-entorno-digital}

\textbf{GNU Stow} y Git transforman la gestión de \textbf{dotfiles} en
un proceso simple y eficiente. Con Stow, centralizas tus
configuraciones, las enlazas con comandos rápidos y las sincronizas con
Git. Ya sea que uses Linux, macOS o WSL, este enfoque te ahorra tiempo y
mantiene tu entorno consistente. ¡Clona tu repositorio, ejecuta
\texttt{stow} y personaliza tu flujo de trabajo hoy! Comparte tus trucos
o configuraciones favoritas en los comentarios.

\section{Publicaciones Similares}\label{publicaciones-similares}

Si te interesó este artículo, te recomendamos que explores otros blogs y
recursos relacionados que pueden ampliar tus conocimientos. Aquí te dejo
algunas sugerencias:

\begin{enumerate}
\def\labelenumi{\arabic{enumi}.}
\tightlist
\item
  \href{https://achalmaedison.netlify.app/tecnologia-seguridad/operating-system/2017-05-21-comandos-de-informacion-windows/index.pdf}{\faIcon{file-pdf}}
  \href{https://achalmaedison.netlify.app/tecnologia-seguridad/operating-system/2017-05-21-comandos-de-informacion-windows}{Comandos
  De Informacion Windows}
\item
  \href{https://achalmaedison.netlify.app/tecnologia-seguridad/operating-system/2019-06-19-adb/index.pdf}{\faIcon{file-pdf}}
  \href{https://achalmaedison.netlify.app/tecnologia-seguridad/operating-system/2019-06-19-adb}{Adb}
\item
  \href{https://achalmaedison.netlify.app/tecnologia-seguridad/operating-system/2021-08-17-limpieza-y-optimizacion-de-pc/index.pdf}{\faIcon{file-pdf}}
  \href{https://achalmaedison.netlify.app/tecnologia-seguridad/operating-system/2021-08-17-limpieza-y-optimizacion-de-pc}{Limpieza
  Y Optimizacion De Pc}
\item
  \href{https://achalmaedison.netlify.app/tecnologia-seguridad/operating-system/2021-10-21-usando-apk-en-windown-11/index.pdf}{\faIcon{file-pdf}}
  \href{https://achalmaedison.netlify.app/tecnologia-seguridad/operating-system/2021-10-21-usando-apk-en-windown-11}{Usando
  Apk En Windown 11}
\item
  \href{https://achalmaedison.netlify.app/tecnologia-seguridad/operating-system/2022-05-12-gestionar-versiones-de-jdk-en-kubuntu/index.pdf}{\faIcon{file-pdf}}
  \href{https://achalmaedison.netlify.app/tecnologia-seguridad/operating-system/2022-05-12-gestionar-versiones-de-jdk-en-kubuntu}{Gestionar
  Versiones De Jdk En Kubuntu}
\item
  \href{https://achalmaedison.netlify.app/tecnologia-seguridad/operating-system/2022-07-21-instalar-tor-browser/index.pdf}{\faIcon{file-pdf}}
  \href{https://achalmaedison.netlify.app/tecnologia-seguridad/operating-system/2022-07-21-instalar-tor-browser}{Instalar
  Tor Browser}
\item
  \href{https://achalmaedison.netlify.app/tecnologia-seguridad/operating-system/2022-08-14-crear-enlaces-duros-o-hard-link-en-linux/index.pdf}{\faIcon{file-pdf}}
  \href{https://achalmaedison.netlify.app/tecnologia-seguridad/operating-system/2022-08-14-crear-enlaces-duros-o-hard-link-en-linux}{Crear
  Enlaces Duros O Hard Link En Linux}
\item
  \href{https://achalmaedison.netlify.app/tecnologia-seguridad/operating-system/2022-09-27-comandos-vim/index.pdf}{\faIcon{file-pdf}}
  \href{https://achalmaedison.netlify.app/tecnologia-seguridad/operating-system/2022-09-27-comandos-vim}{Comandos
  Vim}
\item
  \href{https://achalmaedison.netlify.app/tecnologia-seguridad/operating-system/2023-02-16-guia-de-git-y-github/index.pdf}{\faIcon{file-pdf}}
  \href{https://achalmaedison.netlify.app/tecnologia-seguridad/operating-system/2023-02-16-guia-de-git-y-github}{Guia
  De Git Y Github}
\item
  \href{https://achalmaedison.netlify.app/tecnologia-seguridad/operating-system/2023-05-02-00-primeros-pasos-en-linux/index.pdf}{\faIcon{file-pdf}}
  \href{https://achalmaedison.netlify.app/tecnologia-seguridad/operating-system/2023-05-02-00-primeros-pasos-en-linux}{00
  Primeros Pasos En Linux}
\item
  \href{https://achalmaedison.netlify.app/tecnologia-seguridad/operating-system/2023-06-17-01-introduccion-linux/index.pdf}{\faIcon{file-pdf}}
  \href{https://achalmaedison.netlify.app/tecnologia-seguridad/operating-system/2023-06-17-01-introduccion-linux}{01
  Introduccion Linux}
\item
  \href{https://achalmaedison.netlify.app/tecnologia-seguridad/operating-system/2023-06-18-02-distribuciones-linux/index.pdf}{\faIcon{file-pdf}}
  \href{https://achalmaedison.netlify.app/tecnologia-seguridad/operating-system/2023-06-18-02-distribuciones-linux}{02
  Distribuciones Linux}
\item
  \href{https://achalmaedison.netlify.app/tecnologia-seguridad/operating-system/2023-06-19-03-instalacion-linux/index.pdf}{\faIcon{file-pdf}}
  \href{https://achalmaedison.netlify.app/tecnologia-seguridad/operating-system/2023-06-19-03-instalacion-linux}{03
  Instalacion Linux}
\item
  \href{https://achalmaedison.netlify.app/tecnologia-seguridad/operating-system/2023-06-20-04-administracion-particiones-volumenes/index.pdf}{\faIcon{file-pdf}}
  \href{https://achalmaedison.netlify.app/tecnologia-seguridad/operating-system/2023-06-20-04-administracion-particiones-volumenes}{04
  Administracion Particiones Volumenes}
\item
  \href{https://achalmaedison.netlify.app/tecnologia-seguridad/operating-system/2023-07-01-atajos-de-teclado-y-comandos-para-usar-vim/index.pdf}{\faIcon{file-pdf}}
  \href{https://achalmaedison.netlify.app/tecnologia-seguridad/operating-system/2023-07-01-atajos-de-teclado-y-comandos-para-usar-vim}{Atajos
  De Teclado Y Comandos Para Usar Vim}
\item
  \href{https://achalmaedison.netlify.app/tecnologia-seguridad/operating-system/2024-07-15-instalando-specitify/index.pdf}{\faIcon{file-pdf}}
  \href{https://achalmaedison.netlify.app/tecnologia-seguridad/operating-system/2024-07-15-instalando-specitify}{Instalando
  Specitify}
\end{enumerate}

Esperamos que encuentres estas publicaciones igualmente interesantes y
útiles. ¡Disfruta de la lectura!






\end{document}
