\documentclass[
  doc,
  floatsintext,
  longtable,
  a4paper,
  nolmodern,
  notxfonts,
  notimes,
  colorlinks=true,linkcolor=blue,citecolor=blue,urlcolor=blue]{apa7}

\usepackage{amsmath}
\usepackage{amssymb}

\geometry{inner=1in, outer=1in}
\fancyhfoffset[LE,RO]{0cm}


\usepackage[bidi=default]{babel}
\babelprovide[main,import]{spanish}


\babelfont{rm}[,RawFeature={fallback=mainfontfallback}]{Latin Modern
Roman}
% get rid of language-specific shorthands (see #6817):
\let\LanguageShortHands\languageshorthands
\def\languageshorthands#1{}

\RequirePackage{longtable}
\RequirePackage{threeparttablex}

\makeatletter
\renewcommand{\paragraph}{\@startsection{paragraph}{4}{\parindent}%
	{0\baselineskip \@plus 0.2ex \@minus 0.2ex}%
	{-.5em}%
	{\normalfont\normalsize\bfseries\typesectitle}}

\renewcommand{\subparagraph}[1]{\@startsection{subparagraph}{5}{0.5em}%
	{0\baselineskip \@plus 0.2ex \@minus 0.2ex}%
	{-\z@\relax}%
	{\normalfont\normalsize\bfseries\itshape\hspace{\parindent}{#1}\textit{\addperi}}{\relax}}
\makeatother




\usepackage{longtable, booktabs, multirow, multicol, colortbl, hhline, caption, array, float, xpatch}
\usepackage{subcaption}


\renewcommand\thesubfigure{\Alph{subfigure}}
\setcounter{topnumber}{2}
\setcounter{bottomnumber}{2}
\setcounter{totalnumber}{4}
\renewcommand{\topfraction}{0.85}
\renewcommand{\bottomfraction}{0.85}
\renewcommand{\textfraction}{0.15}
\renewcommand{\floatpagefraction}{0.7}

\usepackage{tcolorbox}
\tcbuselibrary{listings,theorems, breakable, skins}
\usepackage{fontawesome5}

\definecolor{quarto-callout-color}{HTML}{909090}
\definecolor{quarto-callout-note-color}{HTML}{0758E5}
\definecolor{quarto-callout-important-color}{HTML}{CC1914}
\definecolor{quarto-callout-warning-color}{HTML}{EB9113}
\definecolor{quarto-callout-tip-color}{HTML}{00A047}
\definecolor{quarto-callout-caution-color}{HTML}{FC5300}
\definecolor{quarto-callout-color-frame}{HTML}{ACACAC}
\definecolor{quarto-callout-note-color-frame}{HTML}{4582EC}
\definecolor{quarto-callout-important-color-frame}{HTML}{D9534F}
\definecolor{quarto-callout-warning-color-frame}{HTML}{F0AD4E}
\definecolor{quarto-callout-tip-color-frame}{HTML}{02B875}
\definecolor{quarto-callout-caution-color-frame}{HTML}{FD7E14}

%\newlength\Oldarrayrulewidth
%\newlength\Oldtabcolsep


\usepackage{hyperref}



\usepackage{color}
\usepackage{fancyvrb}
\newcommand{\VerbBar}{|}
\newcommand{\VERB}{\Verb[commandchars=\\\{\}]}
\DefineVerbatimEnvironment{Highlighting}{Verbatim}{commandchars=\\\{\}}
% Add ',fontsize=\small' for more characters per line
\usepackage{framed}
\definecolor{shadecolor}{RGB}{241,243,245}
\newenvironment{Shaded}{\begin{snugshade}}{\end{snugshade}}
\newcommand{\AlertTok}[1]{\textcolor[rgb]{0.68,0.00,0.00}{#1}}
\newcommand{\AnnotationTok}[1]{\textcolor[rgb]{0.37,0.37,0.37}{#1}}
\newcommand{\AttributeTok}[1]{\textcolor[rgb]{0.40,0.45,0.13}{#1}}
\newcommand{\BaseNTok}[1]{\textcolor[rgb]{0.68,0.00,0.00}{#1}}
\newcommand{\BuiltInTok}[1]{\textcolor[rgb]{0.00,0.23,0.31}{#1}}
\newcommand{\CharTok}[1]{\textcolor[rgb]{0.13,0.47,0.30}{#1}}
\newcommand{\CommentTok}[1]{\textcolor[rgb]{0.37,0.37,0.37}{#1}}
\newcommand{\CommentVarTok}[1]{\textcolor[rgb]{0.37,0.37,0.37}{\textit{#1}}}
\newcommand{\ConstantTok}[1]{\textcolor[rgb]{0.56,0.35,0.01}{#1}}
\newcommand{\ControlFlowTok}[1]{\textcolor[rgb]{0.00,0.23,0.31}{\textbf{#1}}}
\newcommand{\DataTypeTok}[1]{\textcolor[rgb]{0.68,0.00,0.00}{#1}}
\newcommand{\DecValTok}[1]{\textcolor[rgb]{0.68,0.00,0.00}{#1}}
\newcommand{\DocumentationTok}[1]{\textcolor[rgb]{0.37,0.37,0.37}{\textit{#1}}}
\newcommand{\ErrorTok}[1]{\textcolor[rgb]{0.68,0.00,0.00}{#1}}
\newcommand{\ExtensionTok}[1]{\textcolor[rgb]{0.00,0.23,0.31}{#1}}
\newcommand{\FloatTok}[1]{\textcolor[rgb]{0.68,0.00,0.00}{#1}}
\newcommand{\FunctionTok}[1]{\textcolor[rgb]{0.28,0.35,0.67}{#1}}
\newcommand{\ImportTok}[1]{\textcolor[rgb]{0.00,0.46,0.62}{#1}}
\newcommand{\InformationTok}[1]{\textcolor[rgb]{0.37,0.37,0.37}{#1}}
\newcommand{\KeywordTok}[1]{\textcolor[rgb]{0.00,0.23,0.31}{\textbf{#1}}}
\newcommand{\NormalTok}[1]{\textcolor[rgb]{0.00,0.23,0.31}{#1}}
\newcommand{\OperatorTok}[1]{\textcolor[rgb]{0.37,0.37,0.37}{#1}}
\newcommand{\OtherTok}[1]{\textcolor[rgb]{0.00,0.23,0.31}{#1}}
\newcommand{\PreprocessorTok}[1]{\textcolor[rgb]{0.68,0.00,0.00}{#1}}
\newcommand{\RegionMarkerTok}[1]{\textcolor[rgb]{0.00,0.23,0.31}{#1}}
\newcommand{\SpecialCharTok}[1]{\textcolor[rgb]{0.37,0.37,0.37}{#1}}
\newcommand{\SpecialStringTok}[1]{\textcolor[rgb]{0.13,0.47,0.30}{#1}}
\newcommand{\StringTok}[1]{\textcolor[rgb]{0.13,0.47,0.30}{#1}}
\newcommand{\VariableTok}[1]{\textcolor[rgb]{0.07,0.07,0.07}{#1}}
\newcommand{\VerbatimStringTok}[1]{\textcolor[rgb]{0.13,0.47,0.30}{#1}}
\newcommand{\WarningTok}[1]{\textcolor[rgb]{0.37,0.37,0.37}{\textit{#1}}}

\providecommand{\tightlist}{%
  \setlength{\itemsep}{0pt}\setlength{\parskip}{0pt}}
\usepackage{longtable,booktabs,array}
\usepackage{calc} % for calculating minipage widths
% Correct order of tables after \paragraph or \subparagraph
\usepackage{etoolbox}
\makeatletter
\patchcmd\longtable{\par}{\if@noskipsec\mbox{}\fi\par}{}{}
\makeatother
% Allow footnotes in longtable head/foot
\IfFileExists{footnotehyper.sty}{\usepackage{footnotehyper}}{\usepackage{footnote}}
\makesavenoteenv{longtable}

\usepackage{graphicx}
\makeatletter
\newsavebox\pandoc@box
\newcommand*\pandocbounded[1]{% scales image to fit in text height/width
  \sbox\pandoc@box{#1}%
  \Gscale@div\@tempa{\textheight}{\dimexpr\ht\pandoc@box+\dp\pandoc@box\relax}%
  \Gscale@div\@tempb{\linewidth}{\wd\pandoc@box}%
  \ifdim\@tempb\p@<\@tempa\p@\let\@tempa\@tempb\fi% select the smaller of both
  \ifdim\@tempa\p@<\p@\scalebox{\@tempa}{\usebox\pandoc@box}%
  \else\usebox{\pandoc@box}%
  \fi%
}
% Set default figure placement to htbp
\def\fps@figure{htbp}
\makeatother







\usepackage{fontspec} 

\defaultfontfeatures{Scale=MatchLowercase}
\defaultfontfeatures[\rmfamily]{Ligatures=TeX,Scale=1}

  \setmainfont[,RawFeature={fallback=mainfontfallback}]{Latin Modern
Roman}




\title{Comandos en Windows 11 para Gestionar Redes: Domina comandos en
Windows 11 para gestionar redes, obtener IP pública y más.}


\shorttitle{Editar}


\usepackage{etoolbox}



\ccoppy{\textcopyright~2017}



\author{Edison Achalma}



\affiliation{
{Escuela Profesional de Economía, Universidad Nacional de San Cristóbal
de Huamanga}}




\leftheader{Achalma}

\date{2023-02-16}


\abstract{Primer parrafo de abstrac }

\keywords{keyword1, keyword2}

\authornote{\par{\addORCIDlink{Edison Achalma}{0000-0001-6996-3364}} 
\par{ }
\par{   El autor no tiene conflictos de interés que revelar.    Los
roles de autor se clasificaron utilizando la taxonomía de roles de
colaborador (CRediT; https://credit.niso.org/) de la siguiente
manera:  Edison Achalma:   conceptualización, redacción}
\par{La correspondencia relativa a este artículo debe dirigirse a Edison
Achalma, Email: \href{mailto:elmer.achalma.09@unsch.edu.pe}{elmer.achalma.09@unsch.edu.pe}}
}

\makeatletter
\let\endoldlt\endlongtable
\def\endlongtable{
\hline
\endoldlt
}
\makeatother

\urlstyle{same}



\makeatletter
\@ifpackageloaded{caption}{}{\usepackage{caption}}
\AtBeginDocument{%
\ifdefined\contentsname
  \renewcommand*\contentsname{Tabla de contenidos}
\else
  \newcommand\contentsname{Tabla de contenidos}
\fi
\ifdefined\listfigurename
  \renewcommand*\listfigurename{List of Figures}
\else
  \newcommand\listfigurename{List of Figures}
\fi
\ifdefined\listtablename
  \renewcommand*\listtablename{List of Tables}
\else
  \newcommand\listtablename{List of Tables}
\fi
\ifdefined\figurename
  \renewcommand*\figurename{Figura}
\else
  \newcommand\figurename{Figura}
\fi
\ifdefined\tablename
  \renewcommand*\tablename{Tabla}
\else
  \newcommand\tablename{Tabla}
\fi
}
\@ifpackageloaded{float}{}{\usepackage{float}}
\floatstyle{ruled}
\@ifundefined{c@chapter}{\newfloat{codelisting}{h}{lop}}{\newfloat{codelisting}{h}{lop}[chapter]}
\floatname{codelisting}{Listing}
\newcommand*\listoflistings{\listof{codelisting}{List of Listings}}
\makeatother
\makeatletter
\makeatother
\makeatletter
\@ifpackageloaded{caption}{}{\usepackage{caption}}
\@ifpackageloaded{subcaption}{}{\usepackage{subcaption}}
\makeatother
\makeatletter
\@ifpackageloaded{fontawesome5}{}{\usepackage{fontawesome5}}
\makeatother

% From https://tex.stackexchange.com/a/645996/211326
%%% apa7 doesn't want to add appendix section titles in the toc
%%% let's make it do it
\makeatletter
\xpatchcmd{\appendix}
  {\par}
  {\addcontentsline{toc}{section}{\@currentlabelname}\par}
  {}{}
\makeatother

%% Disable longtable counter
%% https://tex.stackexchange.com/a/248395/211326

\usepackage{etoolbox}

\makeatletter
\patchcmd{\LT@caption}
  {\bgroup}
  {\bgroup\global\LTpatch@captiontrue}
  {}{}
\patchcmd{\longtable}
  {\par}
  {\par\global\LTpatch@captionfalse}
  {}{}
\apptocmd{\endlongtable}
  {\ifLTpatch@caption\else\addtocounter{table}{-1}\fi}
  {}{}
\newif\ifLTpatch@caption
\makeatother

\begin{document}

\maketitle

\hypertarget{toc}{}
\tableofcontents
\newpage
\section[Introduction]{Comandos en Windows 11 para Gestionar Redes}

\setcounter{secnumdepth}{5}

\setlength\LTleft{0pt}


En un mundo hiperconectado, saber gestionar tu red en Windows 11 es una
habilidad esencial. ¿Alguna vez te has quedado sin internet en medio de
un proyecto importante o has necesitado verificar tu conexión
rápidamente? Esta guía actualizada para 2025 te enseña comandos
prácticos para obtener tu IP pública, inspeccionar controladores Wi-Fi y
solucionar problemas de red desde la terminal. Perfecta para usuarios
técnicos, profesionales de TI o cualquiera que quiera optimizar su
sistema, aquí encontrarás pasos claros, ejemplos reales y consejos
avanzados.

Introducción: ¿Por Qué Usar Comandos en Windows 11?

Windows 11, con más de 1.500 millones de usuarios en 2025 según
\href{https://www.statista.com/}{Statista}, ofrece una terminal potente
que va más allá de la interfaz gráfica. Comandos como nslookup y netsh
te dan control directo sobre tu red, revelando detalles que las
herramientas visuales no muestran. En esta guía, exploraremos cómo
usarlos para diagnosticar conexiones, verificar hardware y mejorar el
rendimiento. Entidades como Microsoft y tecnologías como Wi-Fi 6 son
protagonistas, mientras términos como ``IP pública'' o ``controladores
de red'' serán tus aliados. Empecemos.

Comandos Esenciales para Redes en Windows 11

Estos comandos básicos te permiten explorar y gestionar tu red desde la
terminal.

Obtener tu IP Pública

Tu IP pública es tu identidad en internet. Para verla:

\begin{itemize}
\tightlist
\item
  Abre CMD o PowerShell.
\item
  Escribe: nslookup myip.opendns.com resolver1.opendns.com.
\end{itemize}

\textbf{Resultado Ejemplo:}

\begin{Shaded}
\begin{Highlighting}[]
\NormalTok{200.121.132.66}
\end{Highlighting}
\end{Shaded}

OpenDNS, un servicio líder en resolución de DNS, te devuelve tu IP
actual. Ideal para verificar tu conexión externa.

Conocer tu IP Local

Para la IP privada en tu red local:

\begin{itemize}
\tightlist
\item
  Usa ipconfig.
\end{itemize}

\textbf{Resultado Ejemplo:}

\begin{Shaded}
\begin{Highlighting}[]
\NormalTok{Dirección IPv4: 192.168.1.100}
\NormalTok{Máscara de subred: 255.255.255.0}
\NormalTok{Puerta de enlace predeterminada: 192.168.1.1}
\end{Highlighting}
\end{Shaded}

Esto muestra tu configuración interna, útil para ajustes de red.

Otros Comandos Básicos

\begin{itemize}
\tightlist
\item
  ping google.com: Mide la latencia (ej.: 20 ms).
\item
  tracert google.com: Rastrea la ruta de paquetes.
\item
  netstat -an: Lista conexiones activas.
\end{itemize}

\textbf{Tabla: Comandos Rápidos}

\begin{longtable}[]{@{}lll@{}}
\toprule\noalign{}
Comando & Uso & Ejemplo \\
\midrule\noalign{}
\endhead
\bottomrule\noalign{}
\endlastfoot
nslookup & Obtener IP pública & nslookup myip\ldots{} \\
ipconfig & Ver IP local & ipconfig \\
ping & Probar conectividad & ping google.com \\
\end{longtable}

Inspeccionando Controladores Wi-Fi con netsh

El comando netsh wlan show drivers revela detalles técnicos de tu
adaptador Wi-Fi.

Cómo Ejecutarlo

En CMD (como administrador):

\begin{Shaded}
\begin{Highlighting}[]
\NormalTok{netsh wlan show drivers}
\end{Highlighting}
\end{Shaded}

\textbf{Resultado Ejemplo (actualizado 2025):}

\begin{Shaded}
\begin{Highlighting}[]
\NormalTok{Interface name: Wi{-}Fi}
\NormalTok{Driver: Intel(R) Wi{-}Fi 6 AX201}
\NormalTok{Vendor: Intel Corporation}
\NormalTok{Date: 10/15/2024}
\NormalTok{Version: 23.80.1.5}
\NormalTok{Radio types supported: 802.11b 802.11g 802.11n 802.11a 802.11ac 802.11ax}
\NormalTok{FIPS 140{-}2 mode supported: Yes}
\NormalTok{Authentication supported: WPA3{-}Enterprise, WPA2{-}Personal, CCMP, GCMP{-}256}
\NormalTok{Wireless Display Supported: Yes}
\end{Highlighting}
\end{Shaded}

Qué Significa

\begin{itemize}
\tightlist
\item
  \textbf{Driver:} Intel Wi-Fi 6 AX201, compatible con Wi-Fi 6
  (802.11ax), estándar dominante en 2025.
\item
  \textbf{Seguridad:} WPA3 y GCMP-256 aseguran conexiones cifradas.
\item
  \textbf{Soporte:} Miracast y protección de tramas (802.11w) para redes
  modernas.
\end{itemize}

Úsalo para confirmar si tu hardware está listo para redes de alta
velocidad.

Usos Prácticos para Profesionales y Usuarios Técnicos

Estos comandos tienen aplicaciones reales en contextos técnicos:

Soporte Técnico y Diagnóstico

\begin{itemize}
\tightlist
\item
  Usa ping y tracert para identificar caídas de conexión en una red
  empresarial.
\item
  Verifica con netsh wlan show drivers si un cliente necesita actualizar
  su adaptador para Wi-Fi 6.
\end{itemize}

Desarrollo y Pruebas

\begin{itemize}
\tightlist
\item
  Programadores pueden usar ipconfig para configurar entornos de
  desarrollo local y nslookup para probar APIs externas.
\item
  Ejemplo: Asegúrate de que tu servidor local (192.168.1.100) responde
  antes de subir código.
\end{itemize}

Gestión de Redes Domésticas

\begin{itemize}
\tightlist
\item
  Con netstat -an, detecta aplicaciones consumiendo ancho de banda.
\item
  Usa ipconfig /release y ipconfig /renew para refrescar tu IP si hay
  conflictos.
\end{itemize}

\textbf{Imagen:} Terminal Windows 11 \emph{ALT: ``Ejecutando comandos de
red en Windows 11 2025''}

Técnicas Avanzadas para Redes en 2025

\begin{enumerate}
\def\labelenumi{\arabic{enumi}.}
\item
  \textbf{Explorar Redes Disponibles:} Usa netsh wlan show networks para
  listar señales Wi-Fi y elegir la mejor.
\item
  \textbf{Script Automático:} Crea un .bat:

\begin{Shaded}
\begin{Highlighting}[]
\NormalTok{@echo off}
\NormalTok{echo IP Pública: \textgreater{} red{-}info.txt}
\NormalTok{nslookup myip.opendns.com resolver1.opendns.com \textgreater{}\textgreater{} red{-}info.txt}
\NormalTok{echo Controladores Wi{-}Fi: \textgreater{}\textgreater{} red{-}info.txt}
\NormalTok{netsh wlan show drivers \textgreater{}\textgreater{} red{-}info.txt}
\end{Highlighting}
\end{Shaded}

  Ejecuta para guardar datos en un archivo.
\item
  \textbf{TF-IDF:} Integra términos como ``Wi-Fi 6'', ``diagnóstico de
  red'', y ``seguridad inalámbrica'' (basados en el TOP 10).
\item
  \textbf{Schema:} JSON-LD para ``HowTo'':

  json

\begin{Shaded}
\begin{Highlighting}[]
\FunctionTok{\{}
  \DataTypeTok{"@context"}\FunctionTok{:} \StringTok{"https://schema.org"}\FunctionTok{,}
  \DataTypeTok{"@type"}\FunctionTok{:} \StringTok{"HowTo"}\FunctionTok{,}
  \DataTypeTok{"name"}\FunctionTok{:} \StringTok{"Cómo Gestionar Redes en Windows 11"}\FunctionTok{,}
  \DataTypeTok{"step"}\FunctionTok{:} \OtherTok{[}
    \FunctionTok{\{}\DataTypeTok{"@type"}\FunctionTok{:} \StringTok{"HowToStep"}\FunctionTok{,} \DataTypeTok{"text"}\FunctionTok{:} \StringTok{"Abre CMD y usa nslookup para tu IP pública"}\FunctionTok{\}}
  \OtherTok{]}
\FunctionTok{\}}
\end{Highlighting}
\end{Shaded}
\end{enumerate}

Conclusión: Toma el Control de tu Red Hoy

Dominar comandos en Windows 11 como nslookup, ipconfig, y netsh wlan
show drivers te da poder sobre tu red en 2025. Esta guía te equipa con
conocimientos prácticos para diagnosticar problemas, optimizar
conexiones y aplicar soluciones técnicas. Abre tu terminal, prueba estos
comandos, y mejora tu experiencia en Windows 11. ¿Quieres más consejos?
Suscríbete a nuestro boletín o visita
\href{https://microsoft.com/}{microsoft.com} para recursos adicionales.

\section{Publicaciones Similares}\label{publicaciones-similares}

Si te interesó este artículo, te recomendamos que explores otros blogs y
recursos relacionados que pueden ampliar tus conocimientos. Aquí te dejo
algunas sugerencias:

\begin{enumerate}
\def\labelenumi{\arabic{enumi}.}
\tightlist
\item
  \href{https://achalmaedison.netlify.app/tecnologia-seguridad/operating-system/2017-05-21-comandos-de-informacion-windows/index.pdf}{\faIcon{file-pdf}}
  \href{https://achalmaedison.netlify.app/tecnologia-seguridad/operating-system/2017-05-21-comandos-de-informacion-windows}{Comandos
  De Informacion Windows}
\item
  \href{https://achalmaedison.netlify.app/tecnologia-seguridad/operating-system/2019-06-19-adb/index.pdf}{\faIcon{file-pdf}}
  \href{https://achalmaedison.netlify.app/tecnologia-seguridad/operating-system/2019-06-19-adb}{Adb}
\item
  \href{https://achalmaedison.netlify.app/tecnologia-seguridad/operating-system/2021-08-17-limpieza-y-optimizacion-de-pc/index.pdf}{\faIcon{file-pdf}}
  \href{https://achalmaedison.netlify.app/tecnologia-seguridad/operating-system/2021-08-17-limpieza-y-optimizacion-de-pc}{Limpieza
  Y Optimizacion De Pc}
\item
  \href{https://achalmaedison.netlify.app/tecnologia-seguridad/operating-system/2021-10-21-usando-apk-en-windown-11/index.pdf}{\faIcon{file-pdf}}
  \href{https://achalmaedison.netlify.app/tecnologia-seguridad/operating-system/2021-10-21-usando-apk-en-windown-11}{Usando
  Apk En Windown 11}
\item
  \href{https://achalmaedison.netlify.app/tecnologia-seguridad/operating-system/2022-05-12-gestionar-versiones-de-jdk-en-kubuntu/index.pdf}{\faIcon{file-pdf}}
  \href{https://achalmaedison.netlify.app/tecnologia-seguridad/operating-system/2022-05-12-gestionar-versiones-de-jdk-en-kubuntu}{Gestionar
  Versiones De Jdk En Kubuntu}
\item
  \href{https://achalmaedison.netlify.app/tecnologia-seguridad/operating-system/2022-07-21-instalar-tor-browser/index.pdf}{\faIcon{file-pdf}}
  \href{https://achalmaedison.netlify.app/tecnologia-seguridad/operating-system/2022-07-21-instalar-tor-browser}{Instalar
  Tor Browser}
\item
  \href{https://achalmaedison.netlify.app/tecnologia-seguridad/operating-system/2022-08-14-crear-enlaces-duros-o-hard-link-en-linux/index.pdf}{\faIcon{file-pdf}}
  \href{https://achalmaedison.netlify.app/tecnologia-seguridad/operating-system/2022-08-14-crear-enlaces-duros-o-hard-link-en-linux}{Crear
  Enlaces Duros O Hard Link En Linux}
\item
  \href{https://achalmaedison.netlify.app/tecnologia-seguridad/operating-system/2022-09-27-comandos-vim/index.pdf}{\faIcon{file-pdf}}
  \href{https://achalmaedison.netlify.app/tecnologia-seguridad/operating-system/2022-09-27-comandos-vim}{Comandos
  Vim}
\item
  \href{https://achalmaedison.netlify.app/tecnologia-seguridad/operating-system/2023-02-16-guia-de-git-y-github/index.pdf}{\faIcon{file-pdf}}
  \href{https://achalmaedison.netlify.app/tecnologia-seguridad/operating-system/2023-02-16-guia-de-git-y-github}{Guia
  De Git Y Github}
\item
  \href{https://achalmaedison.netlify.app/tecnologia-seguridad/operating-system/2023-05-02-00-primeros-pasos-en-linux/index.pdf}{\faIcon{file-pdf}}
  \href{https://achalmaedison.netlify.app/tecnologia-seguridad/operating-system/2023-05-02-00-primeros-pasos-en-linux}{00
  Primeros Pasos En Linux}
\item
  \href{https://achalmaedison.netlify.app/tecnologia-seguridad/operating-system/2023-06-17-01-introduccion-linux/index.pdf}{\faIcon{file-pdf}}
  \href{https://achalmaedison.netlify.app/tecnologia-seguridad/operating-system/2023-06-17-01-introduccion-linux}{01
  Introduccion Linux}
\item
  \href{https://achalmaedison.netlify.app/tecnologia-seguridad/operating-system/2023-06-18-02-distribuciones-linux/index.pdf}{\faIcon{file-pdf}}
  \href{https://achalmaedison.netlify.app/tecnologia-seguridad/operating-system/2023-06-18-02-distribuciones-linux}{02
  Distribuciones Linux}
\item
  \href{https://achalmaedison.netlify.app/tecnologia-seguridad/operating-system/2023-06-19-03-instalacion-linux/index.pdf}{\faIcon{file-pdf}}
  \href{https://achalmaedison.netlify.app/tecnologia-seguridad/operating-system/2023-06-19-03-instalacion-linux}{03
  Instalacion Linux}
\item
  \href{https://achalmaedison.netlify.app/tecnologia-seguridad/operating-system/2023-06-20-04-administracion-particiones-volumenes/index.pdf}{\faIcon{file-pdf}}
  \href{https://achalmaedison.netlify.app/tecnologia-seguridad/operating-system/2023-06-20-04-administracion-particiones-volumenes}{04
  Administracion Particiones Volumenes}
\item
  \href{https://achalmaedison.netlify.app/tecnologia-seguridad/operating-system/2023-07-01-atajos-de-teclado-y-comandos-para-usar-vim/index.pdf}{\faIcon{file-pdf}}
  \href{https://achalmaedison.netlify.app/tecnologia-seguridad/operating-system/2023-07-01-atajos-de-teclado-y-comandos-para-usar-vim}{Atajos
  De Teclado Y Comandos Para Usar Vim}
\item
  \href{https://achalmaedison.netlify.app/tecnologia-seguridad/operating-system/2024-07-15-instalando-specitify/index.pdf}{\faIcon{file-pdf}}
  \href{https://achalmaedison.netlify.app/tecnologia-seguridad/operating-system/2024-07-15-instalando-specitify}{Instalando
  Specitify}
\end{enumerate}

Esperamos que encuentres estas publicaciones igualmente interesantes y
útiles. ¡Disfruta de la lectura!






\end{document}
